\chapter{Related Work}
\label{chap:related Work}
In this chapter we provide an overview about the current literature for the visualization of large time-oriented business data. 


% Cognitive Fit Theory

% Time-oriented Work: Aigner et al

% Data interaction taxonomies: Shneiderman, Keim
Studies compared animation, small multiples and traces. Yet, they found that none of them is able to scale beyond 200 data items\cite{Robertson2013}. Analyzing large datasets can be enhanced by abstraction. However, the authors found a trade-off between abstraction and accuracy: with low abstraction and a high accuracy there exists the problem of cluttering.

% Visualizing Big Data: Shneiderman, Tufte (reduce information overload), reduce information complexity (Bačić), evtl (Zack, 2007)

% Business Information Visualization: Tegarden, Bačić
\textbf{Business Information Visualization} Tegarden and Bačić\cite{Bacic,Bacic2013,Bacic2012TheCreation,Tegarden1999} published to the topic how business data can be visualized. While Tegarden describes several visualization techniques for Business Information Visualization (BIV) and their applications Bačić explored the process of knowledge creation\cite{Bacic2012}.

% Tool Comparison
Forrester Research, Inc. published a survey\cite{Evelson2012} regarding commercial advanced visualization tools.

In summary, the literature...\todo{einfügen, worüber die Literatur informiert}