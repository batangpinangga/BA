\chapter{Basic Concepts and Terminology}
\label{concepts}
In order to give a clear impression of the used terms and concepts by introducing and defining them in this chapter.\\*
\textbf{Information Visualization}\\*
Information Visualization is the graphical representation of non-spatial or abstract data\cite{Keim}. In contrast to scientific visualization the data which is visualized in information visualization does not have an inherent 2D or 3D structure\cite{Shneiderman2008} and thus, no spatial relation. Usually abstract data comes in data tables with rows and columns. In Information Visualization columns are mapped to graphical attributes such as position, color, size, orientation, texture or hue. 
As business data usually is abstract, discrete and multi-variate\cite{Tegarden1999} the type of visualization for business data is Information Visualization.\\*
\textbf{Business Information Visualization}\label{BIV}\\*
Business Information Visualization(BIV) denotes the visualization of business data. While business data appears in multiple applications BIV usually is achieved with the help of computer tools which range from data loading to interactive visual analysis. Usually, the tool user has few programming knowledge. \\*
\textbf{Visualization Tools}\\*
The set of business tools for BIV ranges from Business Intelligence (BI) and Analytics, visualization, data discovery to data mining. As the differentiation between those tools is not selective and the terms are not clearly distinguished we make the following differentiation:

Data mining tools cover the extraction of patterns and model the underlying data structure\cite{FerreiradeOliveira2003}. When data mining is used together with visualization data mining is based on automated algorithms which detect relevant patterns and display the results afterwards. In contrast, visual data exploration is a completely human guided process\cite{FerreiradeOliveira2003}. First data is displayed on the screen as a visualization and with the help of human visual capabilities new hypothesis are formed. Data visualization is a more general term for generating a graphical representation out of data and is used in BI, Analytics, data discovery and in visual analytics. We will use visualization tool as a tool which supports visual data exploration. 


\iffalse
Data Mining tools allow automatic decision-making by algorithms which are applied to the data and extract patterns in an automatic way\cite{Goebel1999}. Exploratory data analysis (EDA) tools are used to mine data with support of human input. We will use the definition of EDA tools if we speak of visualization tools in this work. As a pwc-survey showed eventhough automatic ways for decision support exist data analysis still relies on human judgement and thus\cite{PwC2016}, visualization tools are used to support the business user in the data discovery process. The main goal of visualization tools is the user support in gaining insights into the data. 
Visualization tools display hundreds of items on the screen and offer interaction techniques such as zooming and filtering\cite{Shneiderman2008}.
\fi
\textbf{Visualization technique}\\*
The way how data variables are mapped to graphical primitives is called visualization technique. To avoid the redundant use of the term visualization we will call visualization techniques simply \textit{techniques} in the following work. Typical examples for techniques are bar charts, line charts or scatterplots. Thereby, every technique has its own characteristic in presenting data. These characteristics include used visualization attributes, the mapping, the use of aggregation methods and dimensionality. Visualization characteristics are called the \textit{visual metaphor} of a technique\cite{Tegarden1999}. Each metaphor has its own strengths and weaknesses and its particular application. In this work we will focus on \textit{time-oriented data} and thus, only consider time-oriented techniques. Moreover, we will not study visualization systems which combine multiple views and interaction techniques.\\*
\textbf{Time-oriented Data}\\*
Data which is linked to time\cite{Aigner2011} is called \textit{time-oriented data}. Time-oriented data has specific characteristics such as linear/cyclic, discrete/continuous or event-based/interval-based. The data type of time-oriented data will be discussed in \ref{data}.\\*
\textbf{Large Data}\\*
Often, time-oriented data sets are very large and multi-variate which makes it difficult to analyze them. The question is how time-oriented data can be analyzed if the number of data points exceeds the screen resolution. This brings us to the definition of large time-oriented data. 
We define large time-oriented data as abstract time-dependent data with a high data volume  which is too large to fit on the screen\cite{Shneiderman2008}. Multi-variate time series are time series where one data item holds several variables at the same point of time\cite{Aigner2011}.
In the following work, we will use time-oriented data equivalently for large time-oriented data. 
\iffalse
Decision-maker usually are part of the management and thus the majority of them uses these tools with small programming knowledge. Therefore, tools have to be self-explaining, easy-to-use \cite{Crapo2000} and without the requirement of extensive programming\label{user}. Visualization plays an important role as it reduces information overload\cite{Keima} and simplifies the process of problem-solving\cite{Zhang}. Eventhough, we only consider visualization tools which are used to explore data visualization tools have two roles of presentation and exploration\cite{Crapo2000}. Visualization as presentation is either used to display data without any data mining algorithm or visualization as presentation is used to present the results of a data mining algorithm. Visualization as exploration is used before and during the data mining algorithm to explore the data interactively. This group is called visual analytics. The decision-maker needs both processes for decision making as results are presented on the screen and to explore the data interactively\cite{Ware2012a}. 
Speaking of visualization an important data type for business is time-oriented data(\ref{data}) as it allows business to analyze the past and predict the future of the company\cite{Ao2010}. We will have a closer look at user tasks in section \ref{tasks}.
\fi










