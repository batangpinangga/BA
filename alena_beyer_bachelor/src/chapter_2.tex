\chapter{Definitions and Related Work}
\label{concepts}
In order to give a clear understanding of the state-of-the art this chapter provides the definition of used concepts and an overview of the current literature that intersects with the topic of this thesis. The relevant research areas can be summarized as business information visualization,  time-oriented data visualization, large scale visualization and surveys of visualization tools. 

\section{Definitions}
\textbf{Information Visualization}:
Information Visualization is the graphical representation of non-spatial or abstract data  \cite{Keim2006}. In contrast to scientific visualization data which is visualized in information visualization does not have an inherent 2D or 3D structure  \cite{Shneiderman2008} and thus, no spatial relation. Abstract data usually exists in data tables with rows and columns. These columns are mapped in Information Visualization to graphical attributes such as position, color, size, orientation, texture or hue. 
As business data usually is abstract, discrete and multivariate  \cite{Tegarden1999} the type of visualization for business data is Information Visualization.\\*

\textbf{Business Information Visualization}\label{BIV}:
The visualization of business data is called Business Information Visualization (BIV). While business data appears in multiple applications \gls{BIV} usually is achieved with the help of computer tools which range from data loading to interactive visual analysis. \\*

\textbf{Visualization Tools}\label{tools}:
Tools for visual data exploration appear under several names which range from Business Intelligence (BI) and Analytics, data discovery, data mining to visualization tools. As the differentiation between those tools is not selective and the terms are not clearly distinguished we make the following differentiation:
Data mining tools cover the extraction of patterns and model the underlying data structure  \cite{FerreiradeOliveira2003}. Hereby, data mining is based on automated algorithms which detect relevant patterns and display the results afterwards statically in terms of reports or visualizations. In contrast, visual data exploration (VDA) is a human guided process  \cite{FerreiradeOliveira2003}. First, data is displayed on the screen as a visualization and with the help of human visual capabilities new hypothesis are formed. In VDA tools the user needs to interact with the data by changing parameters, filtering, zooming, defining new user input. Usually, the user of these tools has little programming knowledge.
\iffalse
Data visualization is a more general term for generating a graphical representation out of data and is used in BI, Analytics, data discovery and in visual analytics.
\fi
Visualization tools can denote both tools to represent data mining results and tools for VDA. We will use visualization tool in terms of a tool which supports visual data exploration. 


\iffalse
Data Mining tools allow automatic decision-making by algorithms which are applied to the data and extract patterns in an automatic way  \cite{Goebel1999}. Exploratory data analysis (EDA) tools are used to mine data with support of human input. We will use the definition of EDA tools for  visualization tools in this work. As a pwc-survey showed eventhough automatic ways for decision support exist data analysis still relies on human judgement and thus  \cite{PwC2016}, visualization tools are used to support the business user in the data discovery process. The main goal of visualization tools is the user support in gaining insights into the data. 
Visualization tools display hundreds of items on the screen and offer interaction techniques such as zooming and filtering  \cite{Shneiderman2008}.
\fi

\textbf{Visualization technique}: The way how data variables are mapped to graphical primitives is called visualization technique. To avoid the redundant use of the term visualization we will call visualization techniques simply \textit{techniques} in the following work. Typical examples for techniques are bar charts, line charts or scatterplots. Thereby, every technique has its own characteristic in presenting data. These characteristics include visualization attributes, the mapping, the use of aggregation methods and dimensionality. Visualization characteristics are called the \textit{visual metaphor} of a technique  \cite{Tegarden1999}. Each metaphor has its own strengths and weaknesses and its particular application. In this work we will focus on \textit{time-oriented data} and thus, only consider time-oriented techniques. Moreover, we will not study visualization systems which denotes techniques requiring a specific software. These are usually publications with proprietary software such as \textit{Time Searcher}  \cite{Hochheiser2004,Buono2005}. This restriction is made to create generalisability.\\*

Besides the visualization terms this work will refer to large, multivariate, time-oriented data. The data type will be discussed in detail in \ref{data}. Short definitions are given in the following: 

\textbf{Time-oriented Data}: Data which is linked to time  \cite{Aigner2011} is called \textit{time-oriented data}. Time-oriented data has specific characteristics such as linear/cyclic, discrete/continuous or event-based/interval-based. \\*

\textbf{Large Data}: We define large time-oriented data as abstract time-dependent data with a high data volume which is too large to fit on the screen  \cite{Shneiderman2008}. 
In the following work, we will use time-oriented data equivalently for large time-oriented data. \\*

\textbf{Multivariate Data}: 
Multivariate time series are time series where one data item holds several variables at the same point of time \cite{Aigner2011}. In the analysis of multivariate time series of different variables and their combinations are interesting. In order to gain understanding of their development over time the challenge of multivariate data is the selection of meaningful dimensions and their visualization. 

\iffalse
Decision-maker usually are part of the management and thus the majority of them uses these tools with small programming knowledge. Therefore, tools have to be self-explaining, easy-to-use  \cite{Crapo2000} and without the requirement of extensive programming\label{user}. Visualization plays an important role as it reduces information overload  \cite{Keima} and simplifies the process of problem-solving  \cite{Zhang}. Eventhough, we only consider visualization tools which are used to explore data visualization tools have two roles of presentation and exploration  \cite{Crapo2000}. Visualization as presentation is either used to display data without any data mining algorithm or visualization as presentation is used to present the results of a data mining algorithm. Visualization as exploration is used before and during the data mining algorithm to explore the data interactively. This group is called visual analytics. The decision-maker needs both processes for decision making as results are presented on the screen and to explore the data interactively  \cite{Ware2012a}. 
Speaking of visualization an important data type for business is time-oriented data(\ref{data}) as it allows business to analyze the past and predict the future of the company  \cite{Ao2010}. We will have a closer look at user tasks in section \ref{tasks}.
\fi


\section{Related Work}
%: Topic Business Information Visualization
Although Information Visualization is intensively researched  \cite{Shneiderman2008,  Shneiderman2002,  Shneiderman1996,  Keim2002} only few researchers published about BIV. The term is defined as the use of visualization technologies to visualize business data  \cite{Tegarden1999}.  Besides the definition Tegarden considered the data types,  the users and the visualizations. Zhang  \cite{Zhang1995,  Zhang1998,  Zhang2001} published a generalized visualization model in which she described the scope of BIV. According to Zhang BIV has to deal with non-geometric data and on the other side consider the human problem-solving process. Bačić and Zhang focused on the business user perspective of problem-solving. This psychological view of the user is explored by Bačić. He studied the process of knowledge creation  \cite{Bacic2012} and how Business Intelligence can support business decision-making  \cite{Bacic2013,  Bacic2012}. A more generalized perspective is covered by Ware \cite{Ware2012a}. He examined how to design information visualization for human perception. The related works to BIV are the foundation for this work's definition of user tasks. 
\par
% Topic: Time-oriented Data visualization
Another important aspect for this work is the visualization of time-oriented data. 
%As time-oriented data appears in the literature with various names a lot of researchers published about time-oriented data. The following works cover different terms of data which is linked to time:  time-dependent  \cite{Mueller2003,  Tominski2005,  Kriglstein2014,  Aigner2007,  VanBuuren2001,  FerreiradeOliveira2003,  Yang2003,  Chung2014,  Rind2011},  time-varying  \cite{Moere2004},  time-oriented  \cite{Aigner2008,  Aigner2007,  Aigner2011,  Hinum2005,  Walker2016} or time-related  \cite{Keim2004}. 
Significant work to the visualization of time-oriented data was done by Aigner \cite{Aigner2011,  Aigner2008,  Aigner2007} who proposed a taxonomy for the time-domain and various visualization techniques. He summarized the key criteria of time-oriented data which influence the visualization. After we analyzed the data characteristics we used the \textit{frame of reference} and the \textit{number of variables} in the selection of visualization techniques. The frame of reference differentiates between abstract and spatial. The number of variables divides data into univariate and multivariate. Speaking of univariate time-oriented data there is a lot of ongoing research called \textit{time series}  \cite{Aigner2011,  Buono2005,  Walker2016,  Leonard2005,  Chen1993,  Esling2012}. 
Other differentiations concerning time were published by Kriglstein et al \cite{Kriglstein2014}. Their hypothesis is that time-oriented data can be presented in two ways: either by \textit{animation} or by using \textit{space-metaphors}. One example for a space-metaphor is the timeline where time is mapped to a line. In their work they collected experimental findings for animation and space-metaphors. These studies compared animation,  small multiples and traces. Yet,  they found that none of them is able to scale beyond 200 data items  \cite{Robertson2013}. Instead they suggested to use temporal abstraction for analyzing large time-oriented data sets which is explained in detail in section \ref{temporalabstraction}.  Aigner et al.'s survey of visualization techniques for time-oriented data provides the basis of studied visualization techniques. 
\par
% Topic: Large scale data visualization
The problem of large scale data visualization is known in literature as
\textit{large}  \cite{PiringerHarald2011,  Keim2001,  Keim1996,  Tennekes2013,  Yang2003,  Keim2005,  Wickham2013}, \textit{large-scale}  \cite{Leonard2005,  PiringerHarald2011,  Cuzzocrea2011,  Keim2005},  \textit{Big Data} \cite{Patil,  Keahey2013,  Chen2012} and \textit{data-intensive}  \cite{PhilipChen2014,  S.MD.Mujeeb2005}.
In the context of Big Data Visualization Wang et al. summarized the current challenges and methods  \cite{Wang2015}. Besides parallelized computation and the handling of unstructured data,  visualization tools are challenged. While some works tackled the problem of reducing data others invented visualization techniques to display as much data as possible without aggregation  \cite{Krzywinski2009,  Luo2012,  Fekete2002}. Important work was published by Keim \cite{Keim1996}. He proposed five categories for visualization techniques: pixel-oriented \cite{Keim1995,  Stein2013,  Keim2000,  Keim1996pixel,  Keim2001,  Keim2005,  Keim2008},  icon-based \cite{Chung2014,  Borgo2013,  Fanea2005},  hierarchical  \cite{Yang2003, Shneiderman1992, LeBlanc1990},  graphic-based and geometric \cite{Noirhomme-Fraiture2002}. Visualizations of large data sets are called \textit{visual scalable} visualizations. Visual or perceptual scalability is defined as the capability of visualization tools to display large data sets in an effective manner  \cite{Eick2002}. Eick described that an ideal measure of visual scalability would be the \textit{number of insights} caused by a visualization tool\cite{Eick2002}. The definition of insights differ from 'Aha' moments to knowledge building. While 'Aha' moments can be measured by neural activity knowledge building is harder to grasp. As Information Visualization uses the term insights in both ways Eick defined an initial step to measure the visual scalability by measuring the scalability of visualizations \textit{(Visualization Characteristics}) as well as the scalability of tools \textit{(Database Metrics)}. 
Another important factor of large data visualization is interaction. Interactions with large databases are covered in  \cite{Buono2005, Jerding1998, Mackinlay1991, Keim2005}. Keim et al.  \cite{Keim2005} cover general interaction techniques while Mackinlay et al. introduce new interaction techniques for large data  \cite{Mackinlay1991}. Buono2005 et al. describe the system \textit{TimeSearcher} and the implemented interaction techniques  \cite{Buono2005}. Visualization, data reduction and interaction techniques are 
\par
% Topic: Survey of visualization tools
In the last chapter we are comparing different visualization tools and their ability to display large data.  Bikakis et al. surveyed different generic visualization systems and graph-based systems in the semantic web  \cite{Bikakis2016}. They compared the spectrum of analytical methods and visualization techniques. Yet,  these systems are usually not used in business. 
Other tool comparisons are published by Zhang et al.  \cite{Zhang2012} and Patil  \cite{Patil}. These works compared commercial visual analytics system in the era of Big Data. In contrast Harger et al. surveyed open source visual analytics systems  \cite{Harger2012}. For BI and Analytics the technology research center Gartner publishes a market overview of the most important Business Intelligence(BI) and Analytics tools which also serve for data discovery every year. The leading tools also appear in the survey  \cite{Evelson2012} regarding commercial advanced visualization tools and the works of Zhang  \cite{Zhang2012}. All the surveys consider visualization,  interaction and analysis capabilities.  
\par
This work will contribute to the visualization of large data by focusing on time-oriented data. We will study different visualization techniques for time-oriented data in business and point out requirements for visualizing large time-oriented data in visualization systems. In a next step,  visualization tools which are used in business,  are selected and the requirements are checked. Finally,  we give a recommendation for the tool use depending on the ease-of-use and programming skills and show further research topics.






