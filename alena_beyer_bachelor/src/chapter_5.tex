\chapter{Future Work}
\label{Future Work}

\section{Limitations}
In Chapter 3 we classified the visualization techniques according to their scalability. Therefore, we made assumptions such as the number of data rows and number of attributes \textit{based on the original paper} when they got published. We did not consider any updates and extensions of the respecting visualization technique. To give an example, the scalability of \textit{TimeWheel} was set on the 2D-TimeWheel although an extended 3D-Version exists. We decided to consider the visualization techniques mentioned in\cite{Aigner2011} as they are recommended for time-oriented data but extensions may have a better scalability.  

In Chapter 4 the success criteria score is based on the assumption that wrinting in a tool specific language is more difficult for a business user than writing in a known language such as Java, javascript or R. This assumption does not assume that a tool specific language might support the user better than only writing in a known language. This limitation has to be considered in the interpretation of our results.
\section{Future Research}


As business users should not be distracted by complex visualization systems \cite{Tegarden1999} evaluating the applicability for business of the proposed visualization techniques would be an important future work.
