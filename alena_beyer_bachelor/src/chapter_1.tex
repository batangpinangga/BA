\chapter{Introduction}
\label{chap:introduction}

\section{Motivation}
In the era of Big Data companies face the problem of visualization of large amounts of data. As data nowadays can easily be collected the data volume continually is growing. Moreover, with the rise of companies as google, Uber and Netflix other business discover the value of data science and strive to do "Big Data". Therefore, their aim is to gain insights into the data by analyzing it. Yet, the data they need to analyze is massive by its volume. In the process of big data analysis information visualization plays an important part: it reduces cognitive overload, complexity and thus, supports the data analyst by visual representations.
\par
Yet, huge amounts of data are a new challenge for data visualization techniques. Beside technical challenges such as fast filtering and aggregation on databases there is the need for displaying billions of database records on a limited screen with a usual screen size of 1600x1200px. With the screen limitation problems such as pixel overlap, visual clutter and information overload arise. Visualization has to overcome the Big Data challenge with appropriate visualization techniques.
As the aim of visualization is insight into data, data visualization must present data in a way so that the reader gains new information out of the depicted visualization.
\par
One important research area in business is the visualization of multivariate time-oriented data. Even though application areas for business data are manifold one commonality is the collection and analysis of time-dependent data. Streaming Data is one famous example for time-dependent data as it denotes continually arriving data marked with a timestamp and usually of massive data volume \cite{o2002streaming} . As streaming data can appear in various domains such as customer data, smart manufacturing, fraud detection, the internet of things or risk management time-oriented data is highly relevant for business.

\section{Research Question}
As time-oriented data is characterized by high volume the question rises whether current tools are able to display big time-oriented data to the business user in an effective manner. The aim of this work is approach the research question above by characterizing the data, studying the business user and different visualization techniques which are appropriate to the characterized data. Based on this analysis requirements for the visualization of large time-oriented data are derived. These requirements are applied in a tool comparison of a small selection of state-of-the-art tools which is used in companies for the analysis of business data.


\section{Structure of this work}
This work is structured in five parts which aim to guide the reader in a logical way through the process of answering the research question. The first part (\ref{concepts}) gives some definitions and basic concepts which are used throughout this work. These definitions are filled with content of already existing publications in the following chapter which summarize related work (\ref{chap:related Work}). The main part of this work are the two succeeding sections with the analysis of large time-oriented data (\ref{chap:BIV}) and the tool comparison(\ref{chap:Tools}). The analysis of large time-oriented data is split up in three parts: the data domain, the user domain and the visualization domain. The data domain covers the question "What is presented?" and studies the particularities of time-oriented data (\ref{data}). The user domain defines psychological findings regarding information visualization  (\ref{perception}) and user tasks for time-oriented data (\ref{tasks}). The visualization domain compares visualization techniques for time-oriented data regarding their ability to display large data (\ref{vis}). Based on this analysis a set of important factors is derived which is the basis for the next section: the tool comparison (\ref{chap:Tools}). For this comparison a set of tool has been selected (\ref{tool:selection}) and compared regarding to their ability to display large data. In a next step a  classification scheme was developed which is based on the factors of the analysis of the preceding chapter (\ref{tool:classification}). This scheme is then applied to the selected tools.
Based on the tool comparison a conclusion was derived.
As this work only covers one specific part of the visualization of large data the last chapter covers limitations of this work (\ref{limitations}) and the resulting future work (\ref{Future Work}).