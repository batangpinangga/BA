\chapter{Introduction}
\label{chap:introduction}

\section{Motivation}
\todo{ersten Outline durch richtige Einleitung ersetzen}
In the era of Big Data huge amounts of data are a new challenge for data visualization techniques. Beside technical challenges such as fast filtering and aggregation on databases there is the need for displaying billions of database records on a limited screen with a usual screen size of 1600x1200px. As the aim of visualization is insight into data, data visualization must present data in a way so that the reader gains new information out of the depicted visualization. 
As more and more companies discover the value of the data and understand how they can use data to unlock hidden mysteries about their customers, internal processes, e.g. to predict the moment when the customer will pay the bill, or production planning, big data must be visualized for the business sector. According to PwC's Global Data and Analytics Survey 2016 30\% of german companies in the medical and descriptive sector use big data in decision making. %(http://www.pwc.com/us/en/advisory-services/data-possibilities/big-decision-survey.html) 
In the current research some solutions exist already to the question how to "squeeze a billion pixels in a limited display" . But non of them compares different tools according to the question, if these tools currently allow to display big data. 
Challenge


\section{Research Question}
This is why the topic of this thesis is the question: "Business Information Visualization of time-oriented data: A comparison of current visualization-tools".
Time-oriented data is only one selected data type out of other data types in information visualization. With the advancing mobile technology and the use of sensors for analytics the analysis of time-oriented data becomes more important for business. Time-oriented data is a well studied research area. Literature for advanced visualizations is widely available. 

\section{Goals of Work}

\section{Structure of Work}
To answer the research question I will identify why advanced visualization methods are needed for displaying time series big data compared to traditional ones.
In defining the difference between traditional and advanced visualization techniques, 
taking psychological results in the perception of "many points" on a screen into account, 
describing the limits of traditional visualization techniques for big data.
Next, I will collect different advanced visualization techniques which make it possible to display big data and reach the above mentioned goal of "Insights in Data". 

Looking at the requirements for the presented advanced visualizations in big data I will build a criteria catalogue after which some selected tools are judged. The criteria catalogue should involve:\\*
Does the tool enables different kinds of advanced visualization types\\*
Allows the tool to implement techniques to realize the research findings about visualizations in big data? (e.g. Is it possible to aggregate visualizations?) \\*
For the test I will select some commercial and open source tools. The criteria to select the tools is not fixed yet. One possibility could be the marketshare and the diversity how the tools are unique compared to other tools.\\*
Last I will apply the criteria to the tools and come to a conclusion and answer to the question above.
