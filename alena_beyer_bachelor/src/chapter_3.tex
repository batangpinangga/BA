\chapter{Business Information Visualization of time-oriented Data}
\label{chap:BIV of BigData}


\iffalse
\listoftodos

\section{Outline} \todo{Remove from BA}
\begin{enumerate}
    \item The role of InfoVis in Business. Visual Analytics. Selfservice. Insights in Company/ Business. $\Rightarrow$ Business Data
    \subitem But: Other data types also in Business: IoT. Not covered because it is a too wide topic.
    \subitem Business use VA tools to get insight into data.
    \item Need for good visualizations for Business data
    \subitem What are business data? $\Rightarrow$ data types
    \item 2nd challenge: BigData.
    \subitem Where does BigData occur? Application Areas. Streaming Data
    \subitem Is BigData relevant for Business Data? Application Areas of Business Data
    \item Solutions in Literature
    \subitem Aggregation
    \subitem Abstraction
    \item Tools in Business
    \subitem Requirements: Visual Analytics Tools
    \subitem New Visualization Techniques -> Extensionability (p.12, "open
framework fed with pluggable visual and analytical components for analyzing
time-oriented data is useful. Such a framework will be able to support multiple
analysis tasks and data characteristics, which is a goal of Visual Analytics."
\cite{Aigner2007})
    \subitem Market Relevance: Qlik, Tableau
    \subitem Other Approaches: Jaspersoft(because its scalable),
    \subitem Comparison
    \item Conclusion: Which tool for time-oriented data?
\end{enumerate}
\fi
----- \\*
% Business Information Visualization
\section{The Role of Visualization in Business}

Tools for Business Intelligence (BI) and Analytics, visualization, data discovery as well as for data mining continually gain importance in companies. Although the differentiation between those tools is not selective all of them are used to gain insight into the company's data and help decision-maker in problem-solving. Decision-maker usually are part of the management domain and thus, come from the marketing, sales or management-field but the majority of them uses these tools without any computer science background. Therefore, tools have to be self-explaining, easy-to-use \cite{Crapo2000} and without the requirement of extensive programming. Especially visualization plays an important role as it reduces the information overload\cite{Keima} and simplifys the process of problem-solving\cite{Zhang}. Talking of visualization for business data the literature uses the term \textit{Business Information Visualization} (BIV). BIV is defined as the use of visualization technologies to visualize business data\cite{Tegarden1999}. 

When tools are used in the discovery of data information visualization has two important roles: presentation and exploration\cite{Crapo2000}. The decision-maker needs both of them as data is presented as a picture on the screen and for solving problems he has to explore the data interactively\cite{Ware2012a}. 
\begin{figure}[H]
    \centering
        \scalebox{.5}{\includegraphics{src/images/VisPipeline}}
    \caption{Visualization Pipeline \cite{Ware2012a}}
    \label{fig:my_label}
\end{figure}

Thus, tools have to present visualization and allow the user to explore the data\cite{Goebel1999,Crapo2000,Ware2012a}.
Nowadays, the challenge in BIV is to handle large amounts of data and displaying them in an effective manner. 

% Users in BIV are usually no computer scientist. [Share of Workers in Business] 
% Self-Service Data
% Role of Visual driven analysis -> need for tools

The effective representation of large amounts of data requires to extend the Overview-Level of the visualization mantra\cite{Shneiderman2008} \textit{Overview first, zoom-in and filter, then details on demand.} by the use of aggregation which can be achieved by appropriate techniques, correct parameterization, interaction and analytical methods\cite{Aigner2008}. Keim summed it up as the Visual Analytics Mantra\cite{Aigner2011}: \textit{Analyze first - Show the Important - Zoom, Filter, Zoom, Filter and Analyze Further- Details on Demand.} 



\section{Definitions}
In order to give a clear impression of the used terms we define frequently used expressions.

\textbf{Information Visualization}\\*
Information Visualization is the graphical representation of non-spatial or abstract data\cite{Keim}. In contrast to scientific visualization the data which is visualized in information visualization does not have an inherent 2D or 3D structure\cite{Shneiderman2008}. 
As business data usually is abstract, discrete and multi-variate according to \cite{Tegarden1999a} the type of visualization for business data is Information Visualization.

\textbf{Visualization technique}\\*
The way how data variables are mapped to graphical primitives is called visualization technique. Typical examples are bar charts, line charts or scatterplots. Thereby, every technique has its own philosophy how to present the data (called \textit{Visual Metaphor}\cite{Tegarden1999}), its own strengths and weaknesses and its particular application. Typically new visualization techniques are created for a specific application and the available amount of visualization technique is huge and continually growing. In this work we will focus on \textit{time-oriented data} and thus, only consider visualization techniques for this cause. 
\\*

\textbf{Visual Data Exploration}\\*

\\*
\textbf{Time-oriented Data}\\*
Time-oriented data is data which is linked to time\cite{Aigner2011}. In literature time-oriented data is also called time-dependent, dynamic, time-variant, time-based, time-varying or temporal data\cite{Moere2004}. Often, time-oriented datasets are very large and multi-variate which makes it difficult to analyze them. The question is how time-oriented data can be analyzed if the number of data points exceeds the screen resolution. This brings us to the definition of Big time-oriented Data.
Big Data in a classical way is defined as high volume, high variety, high varacity and high velocity\cite{PhilipChen2014}.
We define Big time-oriented Data as abstract time-dependent data which is too large to fit on the screen. \cite{Shneiderman2008} 
In the following work, we will use time-oriented Data, time-oriented Big Data and Big time-oriented Data equivalently for Big time-oriented Data.
\cite{Bacic2013} divided time-oriented data into three sections: data with a linear discrete data model, data with a continuous data model and event-based data as described in \cite{Bacic2013} \todo{eventuell noch die Definition von event-based mit aufnehmen}




\textbf{Visualization Tools}\\*
While BI is defined as...\todo{Definition BI} data mining describes the extraction of patterns and models of the underlying data structure\cite{FerreiradeOliveira2003}. When data mining is used together with visualization data mining is based on automated algorithms which detect relevant patterns and display the results afterwards. In contrast, visual data exploration is a completely human guided process\cite{FerreiradeOliveira2003}. First data is displayed on the screen as a visualization and with the help of human visual capabilities new hypothesis are formed. Data visualization is a more general term for generating a graphical representation out of data and is used as well in BI, Analytics, data discovery as in visual analytics. 
Visualization tools display hundreds of items on the screen and offer interaction techniques such as zooming and filtering\cite{Shneiderman2008}.


% Data types for Information Visualization
\section{Data types}
For choosing appropriate visualization techniques the first step is in understanding the underlying data and creating a correct data model\cite{Aigner2011}, so that the visualization technique represents the underlying data structure and offers best insights\cite{Bacic}. \textit{Aigner et. al.} proposed  the following questions to model the visualization problem: 
\begin{itemize}
    \item What is presented?
    \item Why is it presented?
    \item How it is presented?
\end{itemize}
\textbf{What is presented?}\\*
Talking of large time-oriented data for business we will consider the three given characteristics of the data: time-dependency, business data and large size. Business data is collected in many different areas. The following table gives an overview about the applications: 
% insert application table\cite{Tegarden1999a}
The bottom line is that business data ususally is abstract and multi-variate according to Tegarden\cite{Tegarden1999a}. Multi-variate data is usually presented in tables\cite{Borgo2013} and thus we assume that business data is given in tables, each attribute is represented by a column and each row is one data item. The attributes can be either numerical or text-based. 
Talking of time-oriented business data we assume that the data is linked to time. The time-dependency of the data structures the data by a given order. Every data item is mapped to a specific point time with a smallest possible unit such as seconds. Time with a smallest unit is mapped to integer\cite{Aigner2011} and thus we assume that time-oriented business data is discrete. Eventhough, data can be point-based or interval-based (scope), linear or cyclic (arrangement), ordered branching and with multiple perspectives(viewpoint)\cite{Aigner2011}. We will not narrow our focus further but talk on an abstract level. Our perpective is to explore visualization techniques with different scopes, arrangements and viewpoints according to their scalability but of course, every visualization technique is designed for a specific scope such as linearity or seasonal behaviour. The decision for a specific visualization technique is still up to the user. 
Lastely, we will characterize the size of the data. We are considering large and huge amounts of data. Large data is defined according to\cite{Huber1994} as datasets with $10^6$ and huge data with $10^8$ data entries.

% User Tasks
\section{Advanced time-oriented Visualization Tasks}
\textbf{Why is it presented?}\\*
To specify the the problem-domain user task analysis are a common way in the field of software development to describe the tool requirements\cite{Aigner2011}. Different user task taxonomies exist\cite{paterno1997concurtasktrees, Shneiderman2008, Keim2008}. But probably the most famous task taxonomy for visualization is Shneiderman's Visual Information Seeking Mantra \textit{"Overview first, zoom in and filter, then details on demand}. In \cite{Shneiderman2008} Schneiderman extended its own mantra by the use of aggregation markers and Keim formulated it as \textit{Analyze First - Show the Important - Zoom and Filter, and Analyze Further - Details on Demand}\cite{Keima}.
Based on Keims Visual Analytics Mantra there exits the following 6 tasks: 
\todo{Tasks ausformulieren}
\\*
\textbf{T1: Analyze}
\\*
\textbf{T2: Overview}
\\*
\textbf{T3: Zoom}
\\*
\textbf{T4: Filter}
\\*
\textbf{T5: Analyze further}
\\*
\textbf{T6: Details}


\iffalse
\begin{tikzpicture}[sibling distance=12em,
  every node/.style = {shape=rectangle, 
    draw, align=center,
    top color=white}]]
  \node [shape = ellipse] {Visualization Tasks}
    child { node [shape = ellipse] {visualization methods} %time-oriented data: 
      child { node [shape = ellipse] {right visualization method} }
      child { node [shape = ellipse] {right parametrization} 
        child { node {navigation in time} }
        child { node {search} }
        child { node {comparison} }
        child { node {manipulation} } } }
    child { node [shape = ellipse] {analytical methods} %large time-oriented data: 
      child { node {aligned at}
        child { node {relation sign} }
        child { node {several places} }
        child { node {center} } }
      child { node {first left,\\centered,\\last right} } };
\end{tikzpicture}
\fi

% Visualization Techniques
\section{Advanced Visualization of large-scale time-oriented Data}
\textbf{How is it presented?}
To display time-oriented data successfully appropriate techniques
proper parametrization,
interaction facilites are required\cite{Aigner2011}. Additionally, analytical methods such as vertical and horizontal data reduction are necessary to explore large-scale time-oriented data. 

The following section discusses several visualization techniques for time-oriented data and their scalability. We are aware that this discussion cannot be exhaustive as time-oriented data is a current research area and day-to-day new visualization techniques are developed.
Moreover, time-oriented data appears in different areas of business: E-commerce, Smart Health, E-Government, Science \& Technology, Security \& Public safety. Each sector collects different types of data and uses different applications, which makes it impossible to hence to name every single existing visualization technique.
%Different Data Types for time-oriented data

Typical visualization techniques for time-oriented data are 
Static State Replacement,
Time-Series Plots,
Static State Morphing,
Control Applications,
Equilibrium Attainment

 The discussion whether a visualization technique is part of the standard visualization or belongs to advanced visualization is not unambigiously. \textit{Aigner et. al} classify Parallel Coordinates as a standard visualizations\cite{Aigner2011} whereas \textit{Keim et. al.} \cite{Keim} are talking about Parallel Coordinates as a novel techniques. This discussion of course is determined by the time epoche. The longer a visualization technique is available the more it is counted as standard visualization technique. But defining a time-period after which the visualization technique is seen as standard is not possible as other factors influence the judgment of standard or advanced data visualization, such as for example the degree of familiarity. Nevertheless, researcher tried to define advanced data visualization. Russom stated "Advanced Data Visualization (ADV) is able to "scale the visualizations to thousands or millions of data points, can handle different data types and present analytical data structures." \cite{Russom2011}. Thus, we define advanced data visualization as large-scale data visualization techniques which are able to scale to large and huge amounts of data. 
 \\*
\section{Visual Scalability}
%Aspects of Scalability:
% Wie viele Datenpunkte sind notwendig, um Pattern darzustellen? -> data 
% Interaction Techniques
% Downsampling -> Analytical Methods
 The challenges of large-scale data for ADV are \textit{scalability} and \textit{dynamics}\cite{Wang2015}. With its volume the challenges for large-scale data are also challenges for Big Data defined as high volume, high velocity, high veracity and high variety datasets\cite{Wang2015}. In this work we concentrate on the scalability challenge for visualization techniques. The challenge is in finding appropriate techniques\cite{Aigner2008,Keim2005} which scale to large amount of data.
 
Visual or perceptual scalability is defined as the capability of visualization tools in displaying large datasets in an effective manner\cite{Eick2002} such that the user tasks are supported. In the context of time-oriented business data effective means the presentation of patterns to support the temporal analysis tasks. To measure the visual scalability of different visualization techniques for time-oriented data we refer to the work of Eick\cite{Eick2002}. He proposed to measure visual scalability by the database metrics of the dataset and the visual characteristics of the visualization technique.\\*
\textbf{Database metrics} measures the size of the database in bytes, the number of rows or the number of attributes. \\*
\textbf{Visualization characteristics} describe the number of elements and attributes presented on the screen, thus measuring how many distinct items a visualization technique can display.
The combination of the database metrics for the visualization tool and the visualization characteristics for the visualization technique results in the answer how scalable a visualization tool is.

Moreover, besides the database metrics and the visualization characteristic visual scalability is influenced by six factors: 
\begin{itemize}
    \item Human Perception\cite{Keim2005,Deering1998}
    \item Monitor Resolution 
    \item Visual Metaphors
    \item Interactivity
    \item Data structures and algorithms
    \item Computational infrastructure
\end{itemize}
\todo{Entscheiden, inwiefern ich diese 6 Faktoren mit aufnehme. Wenn ich sie mit aufnehme, muss ich sie auch erläutern}

\textbf{Human Perception} 

According to experiments discussing the limits of human perception \cite{Deering1998} the human visual system is able to perceive 15mio pixels per eye. Assuming that the amount of perceivable pixels for two eyes is larger than 15mio pixels but smaller than 30mio pixels due to the overlap of the field of view the max. amount of perceivable pixels (pp) is:
\begin{math}
15 mio. \leq pp < 30 mio.
\end{math}
Nevertheless, the important question is not the amount of perceivable pixels but whether the data structure, patterns, trends and further information in the data can be perceived. To achieve this goal Keim\cite{Keim2005} created \textit{CircleView}.
%Talking of scalability visualization techniques should be able to scale according to their data types, data sources and levels of quality \cite{Keim2008}. 
\textbf{Focal Depth-of-Focus Information}
\\*
\textbf{Monitor Resolution}
Even though large wall-sized screens have been developed in business usually 
\textbf{Visual Metaphors}
Visual Metaphors are the  \todo{Insert definition for visual metaphors}  . Improved visual metaphors enhance the scalability of visualization techniques\cite{Eick2002}. One way to improve visual metaphor are multi-resolution metaphors\cite{Keim2005}. The idea of multi-resolution metaphors is to show less details at the \textit{Overview-Level} of a visualization and more details at the \textit{Detail-Level}. \textit{CircleView} is one visualization technique with a multi-resolution metaphor which clusters data items according to their relevance and displays the clusters first.
\\*

Multi-dimensional visualizations are classified into pixel-oriented, geometric, icon-based, hierarchical and graph-based techniques\cite{Keim2000}.
\textbf{Pixel-oriented} techniques map a data point to a colored pixel. Since each data entry requires one pixel on the monitor pixel-based visualizations can maximal display around 2.000.000. data points. Moreover, they use multiple windows and center the most relevant data in the middle and the less relevant data outside the center\cite{Keim1996}.\\*

\textbf{Icon-based} techniques map each data item onto one icon. The attributes are mapped to different icon features\cite{Keim2001}. In time-oriented visualization InfoBUG and VIE-VIESU belong to the class of icon-based visualizations. As every data item requires one icon icon-based techniques face the challenge of clutter and occlusion\cite{Borgo2013}.


%Data Abstraction for large amount of data
\textbf{Geometric projection} techniques (GP-techniques) map multi-dimensional data to the 2D screen\cite{FerreiradeOliveira2003}. Depending on the visualization technique and its visual methaphor the projection function differs. Often analytical methods are included in the projection and thus, this class of techniques becomes a high-potential class for large datasets as they allow to reduce data horizontally and vertically.
Furthermore, we suggest to devide the GP-techniques into \textit{radial} and \textit{non-radial} visualizations\cite{Diehl2010} as our analysis has shown that a large part of GP-techniques is based on a radial layout. This differentiation is made to analyze the visual scalability of the GP-techniques.\todo{definition of radial wie in \cite{Diehl2010}?}\\*

\begin{table}[th]
	\centering
	% caption format: \caption[<Radial and non-radial GP-techniques>]{<long version>}
	\caption[Table 1]{Radial and non-radial GP-techniques}
	\label{radialTable}
	\begin{tabu}{lcc}
	\toprule
	GP-Technique & radial & non-radial \\
	\midrule
	EventRiver &  & x \\
	Flocking Boids &  & x \\
	Intrusion Detection &  & x \\
	Kiviat Tube & x &  \\
	MultiComb & x &  \\
	Multi-resolution CircleView & x &  \\
	Parallel Glyphs & x &  \\
    Temporal Star & x &  \\
	Time Curves &  & x \\
	Time-tunnel & x & \\
	TimeWheel & x & \\
	Worm Plots &  & x\\
	\bottomrule
	\end{tabu}
\end{table}

Radial layout techniques can scale up to 10-20 attributes. If the mapping is pixel-based it scales up to 1000 pixels\cite{Jayaraman2002}. 
Non-radial techniques

\textbf{EventRiver} was created in journalism to compare hot topics and their relevance over time. This technique uses clustering algorithms to analyze frequent words. Colored Bubbles with different sizes are placed along the x axis according to time. The bubble size represents one cluster and its relevance. The shape shows when the topic appeared and disappeared. Color and the position on the y axis are used to group topics together.
Due to the clustering analysis beforehand the rendering of the visualization data is grouped and thus large data can be displayed. 
EventRiver comes along with interaction techniques such as filtering, reordering and zooming.
\textbf{Flocking Boids} simulate the behaviour of data items in 3D. Thus, data items are represented by a colored, curved line with changing transparency. Based on boid simulation behaviour based rules define the position of the data item over time and its velocity. Different variables can be compared by creating several flocking boids next to each other. Flocking Boid was tested with 12.631 data entries\cite{Moere2004}. 
Analytical Methods such as clustering or subset selection are outsourced to database algorithms and interaction techniques are not implemented but could be extended\cite{Moere2004}.
\textbf{Kiviat Tube} is a unfolded Radar Chart along the z axis in 3D. Several Radar Charts are stacked behind each other along the time (z) axis and form a tube. Thus, variables are mapped on radial aligned planes and can be compared. Interaction such as changing the planes positions and navigating through time enables the user to compare different variables over time.\\*
The number of attributes is limited to approximately 10-20 attributes as the radial layout limits the number of variables. Experiments which measured the maximum number of variables doesn't exist. 
\textbf{MultiComb}
\textbf{Multi-Resolution CircleView} extends the CircleView technique by aggregating data according to their relevance. Similar to CircleView the circle is devided in k segments and k is the number of attributes. The least relevent data is placed at the outer circle with a high aggregation level and the most relevant data in the inner circle. The higher the relevance the lower the aggregation level. The number of displayed data items thus depends on the relevance function. 
\textbf{Parallel Glyphs} pair Parallel Coordinates with Star Glyphs. While similar to Parallel Coordinates each data item is represented by a polyline which connects the vertical axis (attributes) the attribute axis are radially unfolded in 3D and show the data value of the data item over time. Thus, each data value over time is represented by a star glyph. The visualization can be expanded by connection lines over star glyphs. Through the extension of 2D to 3D parallel glyphs are able to display more data rows than parallel coordinates (PC). PC had the problem of clutter while displaying 15.000 data on a gray-scale.  items\cite{Keimb}.
Parallel Glyphs provide brushing of polylines, filtering, axis reordering, rotating in 3 directions, transparency support if the glyphs overlap each other, focus+context presentation through magnification lenses.

\textbf{3D ThemeRiver} (line .... in Table 1) is a 3D representation of the ThemeRiver technique. It inherits the number of usable dimensions and attributes from ThemeRiver, but additionally can map one more variable to the depth of the 3D ThemeRiver.\todo{einzelne Visualisierungen beschreiben}

\textbf{Braided Graph}

\textbf{Spiral Graph}

\cite{Weber2001}

\textbf{Spiral Display}
\cite{Carlis}

\subsection{Analytical Methods}
Comparing every visualization technique the need for data reduction becomes obvious. Since the amount of pixel on a monitor is finite, the space for displaying visualizations is limited. Hence, large datasets of $10^6$ data items cannot be displayed at once but require some kind of preprocessing\cite{FerreiradeOliveira2003,Aigner2011, Keim2005} to reduce data horizontally or vertically. 
Horizontal data reduction is achieved by aggregation algorithms such as Clustering and ANOVA whereas dimensionality reduction reduces data vertically. Therefore, analytical methods such as PCA (Principal Component Analysis) and Data Abstraction are used.
\begin{figure}[H]
    \centering
        \scalebox{.1}{\includegraphics{src/images/dimreduce}}
    \caption{Horizontal Data Reduction}
    \label{fig:my_label}
\end{figure}

\begin{figure}[H]
    \centering
        \scalebox{.1}{\includegraphics{src/images/aggregation}}
    \caption{Vertical Data Reduction}
    \label{fig:my_label}
\end{figure}


%Segmentation Techniques, Factor Analysis, Multidimensional Scaling, FastMap\cite{FerreiradeOliveira2003} Correlation analysis, Information gain or Statistical methods, Sampling, Clustering or Aggregation\cite{Keim2005}



% 3 Methods for Large Data (Aigner2008)
% 1. Temporal Data Abstraction: VIE-VENT + The Spread
% 2. Data Abstraction for Multivariate Data: PCA
% 3. Data Aggregation

\subsection{Temporal Data Abstraction}
The method \textit{Temporal Data Abstraction} is based to the work of \textit{Aigner et al.}\cite{Aigner2011}. The idea is to abstract temporal data by neglecting irrelevant details and focusing on relevant concepts, patterns, shapes over time. 
%For example instead of focusing on the right picture, focusing on the left one. 
\todo{find a fitting example \& add}
%\textit{Horizontal Temporal Abstraction}
%\textit{Vertical Temporal Abstraction}
%Temporal Abstraction in BIV means...



\section{Parameterization}

\section{Need for Interaction Techniques}
Interaction Techniques are a crucial factor for scalable visualization techniques as they enhance visual scalability\cite{Tegarden1999a}. The most important interaction techniques are: 
\subsection{Standard Interaction Techniques}
\textbf{Brushing \& Linking}\\*
The user can select data items on the screen(Brushing) and the respective items are highlighted in every connected window (Linking). Therefore, lasso, rubber-band or rectangular selection enables the user to select groups of data items\cite{Tegarden1999a, Aigner2011}. In the context of time-oriented data a typical brushing activity is the selection of an smaller time-span to see more details during this period of time.
\textbf{Dynamic Queries}\\*
Besides Brushing \& Linking, dynamic queries provide a filter-mechanism by multiple widgets, such as sliders or input fields\cite{Hochheiser2004,Shneiderman2008,Aigner2011}. A specific dynamic query are time-boxes. These boxes are rectangular selection areas which are drawn by the user. The tool then only displays values with a similar pattern to the pattern in the time-boxes.
\textbf{Focus + Context}\\*
\textbf{Fish-eye Views}\\*
\textbf{Table Lens}\\*
\textbf{Magic Lense}\\*
\textbf{Perspective walls\cite{Keim2005}}\\*
\textbf{Zoom + Filter}\\*



\subsection{Advanced Interaction Techniques}
\textbf{Information Mural}
\textbf{Coarse Presentations}



\section{Technical Details}
The used monitors had a screen resolution of 1920*1080 and  2.073.600 px.  
Important is the resolution of the visualization window. (Smaller than 2mio px).
% BigData
