\chapter{Related Work}
\label{chap:related Work}
In this chapter we provide an overview about the current literature which intersects with the topic of our work. Therefore, we consider the four influencing factors: business, time-oriented data, large size and surveys of visualization tools. 
\par
Visualization in the business context is called Business Information Visualization (BIV). This term is defined as the use of visualization technologies to visualize business data\cite{Tegarden1999}. Although Information Visualization is intensively researched\cite{Shneiderman2008,Shneiderman2002,Shneiderman1996,Keim2002} only few researchers published to BIV. Tegarden\cite{Tegarden1999} introduced the most common definition of BIV which was considered above.  Besides the definition BIV includes different aspects regarding the data, the user and the visualization. The psychological view of the user is explored by Bačić. He studied the process of knowledge creation\cite{Bacic2012} and how Business Intelligence can support business decision-making\cite{Bacic,Bacic2012}.  Zhang\cite{Zhang,Zhang1998,Zhang2001} published a generalized visualization model in which she described the scope of BIV. According to her BIV has to deal with non-geometric data and on the other side consider the human problem-solving process. Bačić and Zhang focused on the business user perspective of problem-solving. A more generalized perspective is covered by Ware\cite{Ware2012a}. He examined how to design information visualization for human perception.
\par
Another important aspect for this work is the characteristic of time-oriented data. As time-oriented data appears in the literature with various names a lot of researchers published to time-oriented data. The following works cover different terms of data which is linked to time:  time-dependent\cite{Postfach2003, Tominski2005,Kriglstein2014,Aigner2007,VanBuuren2001,FerreiradeOliveira2003,Yang2003,Chung2014,Rind2011}, time-varying\cite{Moere2004}, time-oriented\cite{Aigner2008,Aigner2007,Aigner2011,Hinum2005,Walker} or time-related\cite{Keimc}. Significant work to the visualization of time-oriented data was done by Aigner\cite{Aigner2011,Aigner2008,Aigner2007} who proposed a taxonomy for the time-domain and various visualization techniques. He summarized the key criteria of time-oriented data which influence the visualization. The \textit{scale} describes whether data is quantitative or qualitative. The \textit{frame of reference} differentiates between abstract and spatial. The \textit{kind of data} classifies data into event-based and state-based and the \textit{number of variables} devides data into univariate and multivariate. These differentiations are used in the classification of visualization techniques.  Speaking of univariate time-oriented data there exists a lot of ongoing research called \textit{time series}\cite{Aigner2011, Buono, Walker,Leonard,Chen1993,Esling2012}.
Other differentiations concerning time were published by Kriglstein et al\cite{Kriglstein2014}. Their hypothesis is that time-oriented data can be presented in two ways: either by \textit{animation} or by using \textit{space-metaphors}. One example for a space-metaphor is the timeline where time is mapped to a line. In their work they collected experimental findings for animation and space-metaphors. These studies compared animation, small multiples and traces. Yet, they found that none of them is able to scale beyond 200 data items\cite{Robertson2013}. Instead they suggested to use abstraction for analyzing large time-oriented data sets. 
\par
Besides the properties of data regarding time, the volume of data is an active research area. This problem is known in literature as
\textit{large}\cite{PiringerHarald2011,Keim2001,Keim1996,tennekes2013visualizing, Yang2003, Keim2005, Wickham2013},\textit{large-scale}\cite{Leonard,PiringerHarald2011,Cuzzocrea,Keim2005}, \textit{Big Data}\cite{Patil,KeaheyUsingData,chen2012business} and \textit{data-intensive}\cite{PhilipChen2014,S.MD.MUJEEB2005}.
While some works tackled the problem of reducing data others invented visualization techniques to display as much data as possible without aggregation\cite{Krzywinski2009,Luo2012,Fekete2002}. Important work was published by Keim\cite{Keim1996}. He proposed five categories for visualization techniques: pixel-oriented\cite{Keim1995,Stein2013,Keim2000,keim1996pixel,Keim2001, Keim2005,Keim2008VisualChallenges}, icon-based\cite{Chung2014,Borgo2013,Fanea2005}, hierarchical\cite{Yang2003,Shneiderman1992,LeBlanc1990}, graphic-based and geometric\cite{Noirhomme-Fraiture2002}. Visualization of large data sets are called \textit{visual scalable} visualizations. Visual or perceptual scalability is defined as the capability of visualization tools of displaying large data sets in an effective manner\cite{Eick2002}. Eick defined a way to measure the visual scalability by measuring the scalability of visualizations \textit{(Visualization Characteristics}) as well as the scalability of tools \textit{(Database Metrics)}. 
Another important factor of large data visualization is interaction. Interactions with large databases are covered in \cite{Buono,Jerding1998,mackinlay1991perspective,Keim2005}. Keim et al.\cite{Keim2005} cover general interaction techniques while Mackinlay et al. introduce new interaction techniques for large data\cite{mackinlay1991perspective}. Buono et al. describe the system \textit{TimeSearcher} and the implemented interaction techniques\cite{Buono}.
\par
In the last chapter we are comparing different visualization tools and their ability to display large data.  Bikakis et al. surveyed different generic visualization systems and graph-based systems in the semantic web\cite{Bikakis2016}. They compared the spectrum of analytical methods and visualization techniques. Yet, these systems are usually not used in business. 
Other tool comparisons are published by Zhang et al.\cite{Zhanga} and Patil\cite{Patil}. These work compared commercial visual analytics system in the era of Big Data. In contrast Harger et al. surveyed open source visual analytics systems\cite{Harger}. For BI and Analytics the technology research center Gartner publishes every year a market overview of the most important Business Intelligence(BI) and Analytics tools which also serve for data discovery. The leading tools also appear in the survey\cite{Evelson2012} regarding commercial advanced visualization tools and the works of \cite{Zhanga}. All the surveys consider visualization, interaction and analysis capabilites.  
\par
This work will contribute to the visualization of large data by focusing on time-oriented data. We will study different visualization techniques for time-oriented data in business and point out requirements for visualizing large time-oriented data in visualization systems. In a next step, visualization tools which are used in business, are selected and the requirements are checked. Finally, we give a recommendation for the tool use depending on the ease-of-use and programming skills and show further research topics.