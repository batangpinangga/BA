\chapter{Introduction}
\label{chap:introduction}
% Start with yout text here.
Welcome to this thesis template!
It is based on the KOMA script classes of \LaTeX{} and should provide you with a basic yet functional template for your bachelor or master thesis.
If you already are familiar with \LaTeX{} you might want to skip the introduction and just go to the formatting checklist right away (see \cref{tab:checklist}).

This template includes the document structure you can use with your thesis and it includes some hints and tips for using \LaTeX{}.
The following examples will assume that you are familiar with writing basic texts in \LaTeX{}, including the overall document structure with chapters and sections etc., creating a PDF file from the source documents and managing your references.
More technical details are integrated into the source document as comments, so if you want change something like the document language start looking into the \texttt{main.tex} file.
In the source files you also will find some comments about the different topics covered in this document.
If you should have any detailed questions concerning a specific package (e.g., you want to create a listing, a complex table or just add some todo notes) it is a good idea to consult the package documentation available at the \emph{\gls{ctan}} homepage\footnote{\url{http://www.ctan.org/pkg/}, accessed on 13.10.2015.}.

If this is your first \LaTeX{} document, you might want to start at the beginning and read some tutorials about how to write a text with it.
For example, \emph{LaTeX Wikibooks}\footnote{\url{http://en.wikibooks.org/wiki/LaTeX/}, accessed on 13.10.2015.} provides a comprehensive overview of how to use \LaTeX{}.
There are tons of other tutorials out in the Internet so you might just search and take one that fits you best.
The \emph{\gls{fim}}\footnote{\url{http://fim.uni-mannheim.de}, accessed on 13.10.2015.} also offers an introductory course and learning material on \LaTeX{}.


\section{Tips and Hints}
Here we start with some dummy elements you might need in your thesis.
I like to start every sentence in a new line in the source document so that it is easier to spot errors reported by \LaTeX.
Use a distinctive name for your labeling scheme so that you can refer to it easily in your text with \texttt{\textbackslash{}cref\{\dots\}}\footnote{I used the mono\hyp{}spaced font here that you can trigger with \texttt{\textbackslash{}texttt\{\dots\}}.}.
I like to use the following scheme:
\begin{itemize}
	\item \texttt{chap:<name>} for chapters
	\item \texttt{sec:<name>} for sections
	\item \texttt{fig:<name>} for figures
	\item \texttt{tab:<name>} for tables
\end{itemize}
You can find more information on the topic at the \emph{LaTeX Wikibooks}\footnote{\url{http://en.wikibooks.org/wiki/LaTeX/Labels_and_Cross-referencing} contains more information on the topic. And this, by the way, is how to insert footnotes and URLs. Accessed on 13.10.2015.}.
As you can see here, to emphasize words use the \texttt{\textbackslash{}emph} command.
You can do this with product names or newly introduced terms.
This is \cref{chap:introduction} of this paper.
Now we can start with some references.
These are typically placed at the end of the sentence or the paragraph they are referring to \citep{Avidan2007,Elalfy2007}.
If you like to name the author of a citation within a sentence you can do it as follows:
In their recent work, \citet{Avidan2007} state that they produced good results with their \gls{seam_carving} approach.
If you refer to online resources such as the article by \citet{Fortune2015}, do not forget to include the access date in the bibliography (see \texttt{library.bib}).

To start a new paragraph just leave (at least) one line blank in the source document.
Always use this approach and do not use manual line breaks (\texttt{\textbackslash{}\textbackslash{}}) as this will mess with \LaTeX{}'s internal formatting rules.
Now we are at acronyms and glossary entries.
If you have acronyms within your text, first add new items to the glossary file.
As soon as you refer to acronyms in your text with the \texttt{\textbackslash{}gls} command, the following will happen:
At the first occurrence, the full term will be displayed.
At any following occurrence just the abbreviated term will get displayed.
This works both for acronyms and for glossary entries.
This is how it looks like:
A different image manipulation approach other than \gls{seam_carving} is the use of \gls{hdr} images.
Such \gls{hdr} images help to record scenes with challenging lighting conditions.
You can also use \texttt{\textbackslash{}Gls} to insert entries starting with a capital letter (e.g., at the beginning of sentences) or \texttt{\textbackslash{}glspl} to insert the plural form.

Maybe you will need to include a table in you paper.
I will just include a very basic table that you can use as a starting point for your own one (see \cref{tab:dummy}).
If you need to create more complex tables again the \emph{LaTeX Wikibooks}\footnote{\url{http://en.wikibooks.org/wiki/LaTeX/Tables}, accessed on 13.10.2015.} gives a more or less complete overview or have a look at \cref{tab:checklist}.
\begin{table}[th]
	\centering
	% caption format: \caption[<short version for list of tables>]{<long version>}
	\caption[Dummy table]{Captions of tables should appear above the table. When integrating a figure, however, the caption should appear below it.}
	\label{tab:dummy}
	\begin{tabu}{ccc}
	\toprule
	 & First & Second \\
	\midrule
	A & 0 & 1 \\
	B & 1 & 0 \\
	\bottomrule
	\end{tabu}
\end{table}

\hyphenation{Su-per-du-per-really-long-word-word-word-word}
In this paragraph I just want so say a few more words on text formatting.
Normally, \LaTeX{} does a proper job with word separation at end of lines.
However, if it does not know words or if they contain special characters, words will exceed the normal text width, resulting in \texttt{overfull} warnings.
For proper handling of special characters like dashes and slashes, you can use \texttt{\textbackslash{}hyp} and \texttt{\textbackslash{}fshyp}.
In that way, words like server\hyp{}based and Client\fshyp{}Server will be treated properly by \LaTeX{}.
There are more forms of dashes.
You can display numeric ranges like 11--15 with en dashes or---if you wish to---insert em dashes for breaks in English sentences.
For German sentences en dashes -- like those -- are used instead.
If you need to explicitly tell \LaTeX{} how words can be separated, you can use the \texttt{\textbackslash{}hyphenation} command by specifying the separation.
In that way, even really long and complicated words like Superduperreallylongwordwordwordword will get separated just as you want them to get separated.
Quotes in \LaTeX{} are not that straight\hyp{}forward either.
To insert quotes in an English text you should do something like ``that''.
In a German text correct quotes are generated by something like \glqq{}this\grqq{} (see comment in source document).
% When you write in German you may also use this format: "`this"'. However, this will not work
% when the English header is active. That is why I used \glqq{}this\grqq{} in the text example.

Now we use this paragraph to insert a dummy figure.
After inserting it, you can refer to it with the \texttt{\textbackslash{}cref} command (see \cref{fig:dummy}) as well.
See how this command automatically included the right type of reference?
If you want to include your own figure, you can basically just copy this example and edit image source, caption and its label.
Refer to the \LaTeX{} source file for an example with subfigures.
You should insert figures at the bottom of the paragraph you are referring to it.
In that way, the figure will get displayed right there or at the top of the next page, if there is not enough space left on the current page.
Luckily, \LaTeX{} will automatically handle all of that for you.
When you take a look in the source document you can see that I inserted a pair of parenthesis (\texttt{\{\}}, an ``empty command'') after the \texttt{\textbackslash{}LaTeX} command.
You will have to do this with certain commands to get the right spacings.
If I just wrote \LaTeX the space between the words would have been wrong (as you can see here).
Also insert the parenthesis if there is no space between a command and the following word.
\begin{figure}[th]
	\centering
	% Image width is set to 0.7 of the whole text width. Change this factor if you need to rescale
	% your image in a different way.
	\includegraphics[width=0.7\textwidth]{src/images/dummy}
	% caption format: \caption[<short version for list of figures>]{<long version>}
	\caption[Dummy figure]{This is just a dummy figure to demonstrate how to include one in your own text.  Source: <source of image if taken from an existing source>}
	\label{fig:dummy}
\end{figure}

% Another example with subfigures.
% \begin{figure}[th]
% 	\centering
% 	\subfloat[First subfigure\label{fig:dummy_subfigs:first}]{\includegraphics[width=0.4\textwidth]{src/images/dummy}}
% 	\hfil
% 	\subfloat[Second subfigure\label{fig:dummy_subfigs:second}]{\includegraphics[width=0.4\textwidth]{src/images/dummy}}
% 	\caption[Dummy figure]{This is just a dummy figure to demonstrate how to include one in your own text.  Source: <source of image if taken from an existing source>}
% 	\label{fig:dummy_subfigs}
% \end{figure}

Ok, that should be enough for now.
If you already know everything stated above you just can remove it and begin to write your own text.
Of course you also can leave it here as a reference until you found everything out.
Good luck with your thesis and have fun with \LaTeX{}!

\section{More Text}
% Just some more random words. Remove it and write your own text.
\Blindtext

