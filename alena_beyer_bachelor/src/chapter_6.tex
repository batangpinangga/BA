\chapter{Future Work}
\label{Future Work}
Throughout this work we identified limitations of this work and remaining challenges for future work which are represented in this chapter. 
\section{Conclusion}\label{conclusion}
In summary, currently there exists no all-in-one solution for the analysis of large time-oriented data in business. While d3.js any possibility regarding visualization and  interaction programming knowledge is required. However, this conflicts with the business approach to do self-service data science. When business decide to use tools such as Tableau, QS and Power BI they need to be aware about the limitations in displaying large data in an effective manner. 


\section{Limitations} \label{limitations}
In Chapter \ref{chap:BIV} we classified the visualization techniques according to their scalability. Therefore, we made assumptions such as the number of data rows and number of attributes \textit{based on the original paper} when they got published. To give an example, the scalability of \textit{TimeWheel} was set on the 2D-TimeWheel although an extended 3D-Version exists. We decided to consider the visualization techniques mentioned in \cite{Aigner2011} as they are recommended for time-oriented data but extensions may have a better scalability.  
Moreover, we took the classification of Aigner et al. \cite{Aigner2011} regarding univariate and multivariate. If the judgement was wrong, our findings are also affected.
In Chapter \ref{chap:Tools} we created the tool criteria score. Therefore, the required programming skills and completeness were considered as criteria. The criterion programming-skills is based on the assumption that writing in a tool specific language is more difficult for a business user than writing in a known language such as Java, JavaScript or R. We assumed that a tool specific language requires more training than a popular programming language. If the tool specific language supports the user better than the popular language our results are affected. This limitation has to be considered in the interpretation of our results. Both criteria were mapped to a numerical scale. The reader might assume that the distance between the numbers are equidistant. This is not true. The scale is an ordinal scale which only positions the tools relative to each other. Moreover, the ranking is \textit{not-weighted} and thus not considering that some features may be more important than others. We decided for a \textit{not-weighted} approach as the weighting depends on the data set and the visualization purpose. 


\section{Future Work}
As this work intersects with various research areas we identified topics which were beyond the scope of this work but needs to be studied in the context of large scale data visualization. 
\textbf{Application depending analysis}\\
In table \ref{table:applications} we described business applications and in section \ref{tasks} user tasks. The described user tasks are abstractly defined user tasks. Yet, each business application has its own specific user tasks and tools. In the context of large data visualization we encourage a application dependent analysis as we think that this analysis can locate specific problems in visualizing huge data amounts. \\*
\textbf{Connection to Cloud}\\
The challenges of visualization for large data also includes the technical challenges such as integration to the cloud and distributed data sets. All of the tools can connect to multiple data sources. In future work, a performance review for the connection of the tools to multiple data sources and to the cloud is recommended. \\*
\textbf{Univariate Data}\\
This work tried to contribute in the visualization of large multivariate time-oriented data. As we excluded univariate data visualization techniques for univariate time-series data were not studied. However, time-series are an important subject of interest in business. Thus, we encourage the evaluation of time-series visualiaztions.\\
\textbf{Evaluation in business}
The techniques considered in this work were all techniques for ADV. Yet, we talked with domain experts which described the problem that business user are unfamiliar with advanced visualization techniques. Tegarden already described that business users should not be distracted by complex visualization systems \cite{Tegarden1999}. That is why evaluating the business applicability of the proposed visualization techniques would be an important future work. 






