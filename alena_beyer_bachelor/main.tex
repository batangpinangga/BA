% LaTeX thesis template.
% Version 1.6
%
% Department of Computer Science IV,
% University of Mannheim,
% Germany
%
% Based on the KOMA script classes.
% Created by Philip Mildner 2013-2015.
% If you have any feedback or if you find errors contact me:
% mildner@informatik.uni-mannheim.de
%
% Relevant class options:
%
% oneside:
% For shorter thesis like seminar papers you can use the single-sided layout, for longer thesis
% the double-sided layout is preferred (and it uses less paper, too). If you would like to have
% a single-sided print, use the 'oneside' option in the documentclass.
%
% BCOR=<value>:
% Depending on your method of binding, you might want to change the BCOR setting to account for
% the part of the pages that are hidden in the cover (e.g., set 'BCOR=10mm' if 1cm is hidden). 
%
% german:
% The standard behavior of this template is to produce English documents. If you would like to
% write your work in German, just include the 'german' option.
%
% draft:
% If you set the 'draft' option, overfull boxes will be highlighted in the PDF file so that you can
% find them more easily.
\documentclass[oneside]{pi4-thesis}

% The title page in this template has a pretty standard layout. While all relevant information are
% displayed on it, it surely could be improved to look nicer. You are invited to change the title
% page to your needs. If you have found a pleasing layout, I will be glad if you share it with me.
%
% Fill in the relevant information in the following lines.
%
% Choose between bachelor thesis or master thesis (or translate it to German).
\piivsubject{Bachelor Thesis\\
\vspace{5mm}
\includegraphics*{src/images/logo}
    }

% The title of your work.
\piivtitle{Business Information Visualization of \\Large Time-Oriented Data}
% Your name.
\piivauthor{Alena Beyer}
% Your matrikel number.
\piivmatrikel{203234}
% Name of your supervisor.
\piivsupervisor{Prof. Dr.-Ing. Bernhard Preim}
\piivsecondsupervisor{Frank Moritz}
% The date you submit your thesis. You can substitute the command with any date.
\date{\today}

% If you want to use the glossary make sure your 'makeindex' toolchain is working correctly.
% Alternetively, you might want to look into the 'xindy' option of the glossaries package.



% -----------------------------------------------------------------------------
% Acronyms Begin
\newacronym{hdr}{HDR}{High Dynamic Range}

\newacronym{ctan}{CTAN}{The Comprehensive \TeX{} Archive Network}

\newacronym{fim}{FIM}{Fachschaft für Mathematik und Informatik}
% Acronyms End
% -----------------------------------------------------------------------------
% -----------------------------------------------------------------------------
% Main Glossary Begin
\newglossaryentry{BIV}
 {
   name=BIV,
   description={Business Information Visualization}
}
\newglossaryentry{QS}
 {
   name=QS,
   description={Qlik Sense}
}
\newglossaryentry{BI}
 {
   name=BI,
   description={Business Intelligence}
}
\newglossaryentry{VDA}
 {
   name=VDA,
   description={Visual Data Exploration}
}
\newglossaryentry{GP}
 {
   name=GP-techniques,
   description={Geometric-Projective Visualization Techniques}
}
\newglossaryentry{PC}
 {
   name=PC,
   description={Parallel Coordinates}
}
\newglossaryentry{ADV}
 {
   name=ADV,
   description={Advanced Data Visualization}
}
\newglossaryentry{DE}
 {
   name=DE,
   description={Display Efficiency}
}
\newglossaryentry{SP}
 {
   name=SP,
   description={Screen Pixels}
}
\newglossaryentry{TBP}
 {
   name=TBP,
   description={Total Amount of Brain Pixels which are stimulated by the Screen pixels}
}
\newglossaryentry{USBP}
 {
   name=USBP,
   description={Unique stimulated Brain Pixels}
}
\newglossaryentry{PP}
 {
   name=PP,
   description={Perceivable Pixels}
}
\newglossaryentry{PCA}
 {
   name=PCA,
   description={Principal Component Analysis}
}
\newglossaryentry{SOM}{
    name=SOM,
    description={Self Organizing Maps}
    }
\newglossaryentry{NLP}
 {
   name=NLP,
   description={Natural Language Processing}
}
\newglossaryentry{GUI}
 {
   name=GUI,
   description={Graphical User Interface}
}
\newglossaryentry{DOM}
 {
   name=DOM,
   description={Document Object Model}
}
\newglossaryentry{TCS}
 {
   name=TCS,
   description={Tool Criteria Score}
}
\newglossaryentry{Q+A}
{
    name=Q\&A,
    description={Question and Answer}
}
\newglossaryentry{SE}
{
    name=SE,
    description={Set Expressions}
}
\newglossaryentry{MiC}
{
    name=MiC,
    description={Mini Charts}
    }
\newglossaryentry{SDC}
{
    name=SDC,
    description={Smart Data Compression}
}
\newglossaryentry{DEx}
{
    name=DE,
    description={Data Extracts}
}
\newglossaryentry{E}
{
    name=E,
    description={Extensions}
}
\newglossaryentry{L}
{
    name=L,
    description={Libraries}
}
\newglossaryentry{P}
{
    name=P,
    description={Programmable}
}
\newglossaryentry{J}
{
    name=J,
    description={Joins}
}
\newglossaryentry{CD}
{
    name=CD,
    description={Calculated Dimensions}
}
\newglossaryentry{DL}{
    name=DL,
    description={Dimension Limitations}
}
\newglossaryentry{H}{
    name=H,
    description={Hide Columns in Data Extracts}
}
\newglossaryentry{CC}
{
    name= CC,
    description={Calculated Columns}
}
\newglossaryentry{CF}
{
    name= CF,
    description={Calculated Fields}
}
\newglossaryentry{DDR}
{
    name= DDR,
    description={Dynamic Data Reduction}
}
\newglossaryentry{DQ}
{
    name= DQ,
    description={Dynamic Query Filters}
}
\newglossaryentry{B}
{
    name= B,
    description={Brushing}
}
\newglossaryentry{V}
{
    name= V,
    description={Views}
}
\newglossaryentry{A}
{
    name= A ,
    description={Aggregation}
}
\newglossaryentry{T}
{
    name= T ,
    description={Trendline}
}
\newglossaryentry{MC}
{
    name=  MC,
    description={Marker Clustering}
}
\newglossaryentry{MM}
{
    name= MM,
    description={Min-Max-Functions}
}
\newglossaryentry{F}
{
    name= F ,
    description={Forecasting}
}
\newglossaryentry{PT}
{
    name= PT ,
    description={Perpective Transformation}
}
\newglossaryentry{N}
{
    name= N,
    description={Natively Integrated in the visualization tool }
}
\newglossaryentry{O}
{
    name= O ,
    description={Omit rows in SQL-Script}
}
\newglossaryentry{FD}
{
    name= FD ,
    description={Fish-eye Distortion}
}
\newglossaryentry{SS}
{
    name= SS ,
    description={Smart Search}
}
% Main Glossary End
% -----------------------------------------------------------------------------
\makenoidxglossaries

\setcitestyle{square}
\setlength{\parindent}{0em}
\setlength{\parskip}{1em}


\begin{document}
%\newcommand{\par}{\paragraph}
\definecolor{qlikgreen}{rgb}{0.3,0.6,0.03}
\definecolor{applegreen}{HTML}{F0F8FF}
\definecolor{lightblue}{HTML}{87CEFA}
\definecolor{d3orange}{rgb}{0.96,0.5,0.3}
%\afterpage{\null\newpage}
\chapter{Acknowledgement}
\setlength{\parindent}{0cm}
To \textit{Elias} - who opened my eyes what really counts in life! Many Thanks!
\par
I am grateful to my parents and my sisters for her emotional and moral support throughout my life. I am also grateful to other family members and friends who supported me along my way.
\par
This thesis has been written as a partial fulfillment requirement to obtain the Bachelor of Science degree at Otto-von-Guericke University, Magdeburg. 
In this context I would like to thank my thesis supervisor \textit{Bernhard Preim} for his support throughout this thesis. Whenever I had questions his replies almost appeared instantly.  
\par
A special mention to my advisor \textit{Frank Moritz}. His regular feedback and input were a constant source of inspiration. With his open mind he helped me in reviewing my work and to take a step beyond the known horizon. Many thanks!
 \par
I am also grateful to the valuable feedback of \textit{Lena Cibulski, Karl Schriek, Vanessa Walter and Birgitta Wollherr} who helped in polishing this work.  
\par
And finally, last but not least a very special thanks to all the staff of Alexander Thamm GmbH. They were always keen to know what I am doing, motivated and encouraged me.     

\begin{figure}[b]
    \includegraphics[width=0.6\textwidth]{src/images/logo}\\
    \includegraphics[width=0.6\textwidth]{src/images/logo-alexanderthamm-gmbh-3}
    \label{fig:my_label}
\end{figure}

\markboth{}{}
%\afterpage{\blankpage}
\afterpage{\null\newpage}

% Abstract is optional. If you do not use an abstract, remove it.
% ---------------------------------
% Begin of abstract
\abstractchap
Information Visualization is widely used to support data analysis. Especially, interactive visual data exploration supports the analyst in gaining insights from the data. Also companies discovered the value of visual data exploration for improving their business by understanding their data. Some examples are understanding customers, predicting machine maintenance or forecasting the business development. Even though industries collect different kinds of data business data usually share the commonality of time-linkage - the so called time-oriented data. Time-oriented data is challenging as this data type is usually of high volume which causes similar problems as Big Data. The visualization of Big Data is problematic because visual clutter, information and orientation loss appear. These problems complicate visual data analysis. 
\par
In literature exist various clutter reduction techniques such as aggregation or multi-resolution. Different visualization techniques scale better and reduce visual clutter. In the context of time-oriented data this thesis analyzes multivariate time-oriented visualization techniques and determines their scalability. As scalable visualization techniques alone are not sufficient data reduction methods and interaction techniques are reviewed to reduce Big Data challenges in visualization. After the analysis of data reduction techniques a criteria catalogue is suggested for displaying large scale data effectively. 
\par
This schema is applied to current visualization tools and frameworks used in business. Data companies make use of visualization tools and frameworks to visually explore their data. They expect to have a tool with many features or an easy-to-use tool. Often, they expect both completeness and ease-of-use.  In order to answer the question of feature support three tools and one framework have been studied with respect to completeness and ease-of-use. 

\markboth{}{}

\afterpage{\null\newpage}
% End of abstract
% ---------------------------------

% ---------------------------------
% Begin of listings
%\microtypesetup{protrusion=false} % disables protrusion locally in the document
\tableofcontents
% If you should not have any figures, tables or acronyms in your paper remove the according list.
\listoffigures
\listoftables

% Uncomment the next line if you use listings in your document.
% \lstlistoflistings
\microtypesetup{protrusion=true} % enables protrusion


% End of listings
% ---------------------------------

% ---------------------------------
% Begin of main part
\mainmatter

\chapter{Introduction}
\label{chap:introduction}

\section{Motivation}
% I: Importance of information visualization
Information Visualization as a method to analyze data has been popular for some time now. At first only a research area, it is now used in many application areas. Likewise companies are discovering the value of data analysis and visualization. The aim is use data analysis to better understand their businesses. In particular, interactive visual data exploration - also called data discovery - has gained importance, as the analyst can not only display data but interactively adapt views and parameters according to his needs\footnote{In the BARC BI Trend Monitor 2017 Data discovery and visualization is ranked most important (7,2 from 10) by 2,800 users, companies and software vendors \cite{Bange2016}.}.
\par
% I: Multi-variate data is 
One important research area for companies is the visualization of multivariate time-oriented data. Even though application areas for business data are manifold, one commonality is the collection and analysis of time-dependent data. Streaming Data is one famous example of time-dependent data as it denotes continually arriving data (marked with a timestamp), which is usually of massive data volume  \cite{Callaghan2002}. As streaming data can appear in various domains such as customer data,  smart manufacturing,  fraud detection,  the internet of things or risk management time-oriented data is highly relevant for business. 
\par 
% I: Tools
Companies usually analyze their data with the help of visualization tools or frameworks. These tools allow users for exploring their data interactively with the help of visualization. These tools promote themselves as self-service data science tools. They promise that anyone would be able to perform data analysis - no programming skills are required. Such tools offer all the necessities for visual analysis. Frameworks however require more programming skills but offer more features.
\par
% I: challenges by Big Data
Yet, interactive visual analysis in the Big Data era is challenging. As sensors can collect data automatically and mobile devices are popular, massive amounts of data are collected in business application areas such as smart phones, machines or cars
\footnote{While out of 1000 surveyed companies only 5.4\% of the firms invested more than \$50MM in Big Data in 2014,  this share rose to 26, 8\% in 2017  \cite{Bean2016}.}. When this data is visually explored the size of the data leads to problems. With a limited screen resolution the data volume greatly exceeds the number of screen pixels, causing visual clutter. Visual clutter in turn hinders the data analyst in finding patterns, in discerning objects and in abstracting data to clusters, trends or correlations. 
\par
% I: Example of Fault Detection
As an example from the automobile industry: modern  vehicles typically perform ongoing collection of data via the car's sensors. These data sets are then used for various purposes such as error detection. If the motor oil sensor measures that the oil is low a corresponding signal will automatically appear in the electronic instrument cluster. If the mileage is above a determined level the driver is requested to visit the garage. All these error signals are emitted from the error memory in the car which is manually implemented during the vehicle's development. However, during development, engineers are typically not yet familiar with error thresholds. Therefore, visual data exploration is a method to detect outliers and determine threshold values. Yet, visualization in this context is challenging as the car's sensors emit signals with very high frequency and a resulting massive data volume. Where large scale visualizations exist, visual exploration can support developers by revealing patterns and outliers. Thus,  the need for scalable visualizations and appropriate tools becomes obvious. 
In order to reduce the visual clutter visualization techniques and visualization tools need to adapt to large data sets and offer appropriate methods to present data.
\par

\section{Research Question}\label{research}
Thus, the research question of this work is the visualization of large data sets which overcomes perceptual limitations and how current visualization tools and frameworks support it. This work focuses on time-oriented data. 
As time-oriented data is characterized by high volume the question arises whether the tools used in business are able to display large time-oriented data sets to the business user in an effective manner. The term effectiveness is defined in \ref{effective}. 
We suppose that visualization tools and frameworks require programming knowledge.
The aim of this work is to approach the research question above by characterizing the data, by studying the business user and different visualization techniques. Based on this analysis, requirements for the visualization of large time-oriented data are derived. These requirements are applied in the comparison of a selection of state-of-the-art tools that are used in companies for the analysis of business data.
\section{Structure of this work}
Chapter \ref{chap:concepts} includes definitions that are used throughout this thesis and  summarises the existing publications that relate to these definitions. In chapter \ref{chap:basics} the characteristics of time-oriented data are analyzed. The main part of this work is dealt with in the two succeeding sections with the visualization of large scaled data (chapter \ref{chap:scalability}) and the tool comparison (chapter \ref{chap:Tools}). 
Based on the tool discussion a conclusion is derived in chapter \ref{chap:conclusion}.
As this work only covers one specific part of visual exploration of large data chapter \ref{chap:conclusion} covers limitations of this work and resulting future work.


\chapter{Definitions and Related Work}
\label{concepts}
In order to give a clear understanding of the state-of-the art this chapter provides the definition of used concepts and an overview of the current literature that intersects with the topic of this thesis. The relevant research areas can be summarized as business information visualization,  time-oriented data visualization, large scale visualization and surveys of visualization tools. 

\section{Definitions}
\textbf{Information Visualization}:
Information Visualization is the graphical representation of non-spatial or abstract data  \cite{Keim2006}. In contrast to scientific visualization data which is visualized in information visualization does not have an inherent 2D or 3D structure  \cite{Shneiderman2008} and thus, no spatial relation. Abstract data usually exists in data tables with rows and columns. These columns are mapped in Information Visualization to graphical attributes such as position, color, size, orientation, texture or hue. 
As business data usually is abstract, discrete and multivariate  \cite{Tegarden1999} the type of visualization for business data is Information Visualization.\\*

\textbf{Business Information Visualization}\label{BIV}:
The visualization of business data is called Business Information Visualization (BIV). While business data appears in multiple applications \gls{BIV} usually is achieved with the help of computer tools which range from data loading to interactive visual analysis. \\*

\textbf{Visualization Tools}\label{tools}:
Tools for visual data exploration appear under several names which range from Business Intelligence (BI) and Analytics, data discovery, data mining to visualization tools. As the differentiation between those tools is not selective and the terms are not clearly distinguished we make the following differentiation:
Data mining tools cover the extraction of patterns and model the underlying data structure  \cite{FerreiradeOliveira2003}. Hereby, data mining is based on automated algorithms which detect relevant patterns and display the results afterwards statically in terms of reports or visualizations. In contrast, visual data exploration (VDA) is a human guided process  \cite{FerreiradeOliveira2003}. First, data is displayed on the screen as a visualization and with the help of human visual capabilities new hypothesis are formed. In VDA tools the user needs to interact with the data by changing parameters, filtering, zooming, defining new user input. Usually, the user of these tools has little programming knowledge.
\iffalse
Data visualization is a more general term for generating a graphical representation out of data and is used in BI, Analytics, data discovery and in visual analytics.
\fi
Visualization tools can denote both tools to represent data mining results and tools for VDA. We will use visualization tool in terms of a tool which supports visual data exploration. 


\iffalse
Data Mining tools allow automatic decision-making by algorithms which are applied to the data and extract patterns in an automatic way  \cite{Goebel1999}. Exploratory data analysis (EDA) tools are used to mine data with support of human input. We will use the definition of EDA tools for  visualization tools in this work. As a pwc-survey showed eventhough automatic ways for decision support exist data analysis still relies on human judgement and thus  \cite{PwC2016}, visualization tools are used to support the business user in the data discovery process. The main goal of visualization tools is the user support in gaining insights into the data. 
Visualization tools display hundreds of items on the screen and offer interaction techniques such as zooming and filtering  \cite{Shneiderman2008}.
\fi

\textbf{Visualization technique}: The way how data variables are mapped to graphical primitives is called visualization technique. To avoid the redundant use of the term visualization we will call visualization techniques simply \textit{techniques} in the following work. Typical examples for techniques are bar charts, line charts or scatterplots. Thereby, every technique has its own characteristic in presenting data. These characteristics include visualization attributes, the mapping, the use of aggregation methods and dimensionality. Visualization characteristics are called the \textit{visual metaphor} of a technique  \cite{Tegarden1999}. Each metaphor has its own strengths and weaknesses and its particular application. In this work we will focus on \textit{time-oriented data} and thus, only consider time-oriented techniques. Moreover, we will not study visualization systems which denotes techniques requiring a specific software. These are usually publications with proprietary software such as \textit{Time Searcher}  \cite{Hochheiser2004,Buono2005}. This restriction is made to create generalisability.\\*

Besides the visualization terms this work will refer to large, multivariate, time-oriented data. The data type will be discussed in detail in \ref{data}. Short definitions are given in the following: 

\textbf{Time-oriented Data}: Data which is linked to time  \cite{Aigner2011} is called \textit{time-oriented data}. Time-oriented data has specific characteristics such as linear/cyclic, discrete/continuous or event-based/interval-based. \\*

\textbf{Large Data}: We define large time-oriented data as abstract time-dependent data with a high data volume which is too large to fit on the screen  \cite{Shneiderman2008}. 
In the following work, we will use time-oriented data equivalently for large time-oriented data. \\*

\textbf{Multivariate Data}: 
Multivariate time series are time series where one data item holds several variables at the same point of time \cite{Aigner2011}. In the analysis of multivariate time series of different variables and their combinations are interesting. In order to gain understanding of their development over time the challenge of multivariate data is the selection of meaningful dimensions and their visualization. 

\iffalse
Decision-maker usually are part of the management and thus the majority of them uses these tools with small programming knowledge. Therefore, tools have to be self-explaining, easy-to-use  \cite{Crapo2000} and without the requirement of extensive programming\label{user}. Visualization plays an important role as it reduces information overload  \cite{Keima} and simplifies the process of problem-solving  \cite{Zhang}. Eventhough, we only consider visualization tools which are used to explore data visualization tools have two roles of presentation and exploration  \cite{Crapo2000}. Visualization as presentation is either used to display data without any data mining algorithm or visualization as presentation is used to present the results of a data mining algorithm. Visualization as exploration is used before and during the data mining algorithm to explore the data interactively. This group is called visual analytics. The decision-maker needs both processes for decision making as results are presented on the screen and to explore the data interactively  \cite{Ware2012a}. 
Speaking of visualization an important data type for business is time-oriented data(\ref{data}) as it allows business to analyze the past and predict the future of the company  \cite{Ao2010}. We will have a closer look at user tasks in section \ref{tasks}.
\fi


\section{Related Work}
%: Topic Business Information Visualization
Although Information Visualization is intensively researched  \cite{Shneiderman2008,  Shneiderman2002,  Shneiderman1996,  Keim2002} only few researchers published about BIV. The term is defined as the use of visualization technologies to visualize business data  \cite{Tegarden1999}.  Besides the definition Tegarden considered the data types,  the users and the visualizations. Zhang  \cite{Zhang1995,  Zhang1998,  Zhang2001} published a generalized visualization model in which she described the scope of BIV. According to Zhang BIV has to deal with non-geometric data and on the other side consider the human problem-solving process. Bačić and Zhang focused on the business user perspective of problem-solving. This psychological view of the user is explored by Bačić. He studied the process of knowledge creation  \cite{Bacic2012} and how Business Intelligence can support business decision-making  \cite{Bacic2013,  Bacic2012}. A more generalized perspective is covered by Ware \cite{Ware2012a}. He examined how to design information visualization for human perception. The related works to BIV are the foundation for this work's definition of user tasks. 
\par
% Topic: Time-oriented Data visualization
Another important aspect for this work is the visualization of time-oriented data. 
%As time-oriented data appears in the literature with various names a lot of researchers published about time-oriented data. The following works cover different terms of data which is linked to time:  time-dependent  \cite{Mueller2003,  Tominski2005,  Kriglstein2014,  Aigner2007,  VanBuuren2001,  FerreiradeOliveira2003,  Yang2003,  Chung2014,  Rind2011},  time-varying  \cite{Moere2004},  time-oriented  \cite{Aigner2008,  Aigner2007,  Aigner2011,  Hinum2005,  Walker2016} or time-related  \cite{Keim2004}. 
Significant work to the visualization of time-oriented data was done by Aigner \cite{Aigner2011,  Aigner2008,  Aigner2007} who proposed a taxonomy for the time-domain and various visualization techniques. He summarized the key criteria of time-oriented data which influence the visualization. After we analyzed the data characteristics we used the \textit{frame of reference} and the \textit{number of variables} in the selection of visualization techniques. The frame of reference differentiates between abstract and spatial. The number of variables divides data into univariate and multivariate. Speaking of univariate time-oriented data there is a lot of ongoing research called \textit{time series}  \cite{Aigner2011,  Buono2005,  Walker2016,  Leonard2005,  Chen1993,  Esling2012}. 
Other differentiations concerning time were published by Kriglstein et al \cite{Kriglstein2014}. Their hypothesis is that time-oriented data can be presented in two ways: either by \textit{animation} or by using \textit{space-metaphors}. One example for a space-metaphor is the timeline where time is mapped to a line. In their work they collected experimental findings for animation and space-metaphors. These studies compared animation,  small multiples and traces. Yet,  they found that none of them is able to scale beyond 200 data items  \cite{Robertson2013}. Instead they suggested to use temporal abstraction for analyzing large time-oriented data sets which is explained in detail in section \ref{temporalabstraction}.  Aigner et al.'s survey of visualization techniques for time-oriented data provides the basis of studied visualization techniques. 
\par
% Topic: Large scale data visualization
The problem of large scale data visualization is known in literature as
\textit{large}  \cite{PiringerHarald2011,  Keim2001,  Keim1996,  Tennekes2013,  Yang2003,  Keim2005,  Wickham2013}, \textit{large-scale}  \cite{Leonard2005,  PiringerHarald2011,  Cuzzocrea2011,  Keim2005},  \textit{Big Data} \cite{Patil,  Keahey2013,  Chen2012} and \textit{data-intensive}  \cite{PhilipChen2014,  S.MD.Mujeeb2005}.
In the context of Big Data Visualization Wang et al. summarized the current challenges and methods  \cite{Wang2015}. Besides parallelized computation and the handling of unstructured data,  visualization tools are challenged. While some works tackled the problem of reducing data others invented visualization techniques to display as much data as possible without aggregation  \cite{Krzywinski2009,  Luo2012,  Fekete2002}. Important work was published by Keim \cite{Keim1996}. He proposed five categories for visualization techniques: pixel-oriented \cite{Keim1995,  Stein2013,  Keim2000,  Keim1996pixel,  Keim2001,  Keim2005,  Keim2008},  icon-based \cite{Chung2014,  Borgo2013,  Fanea2005},  hierarchical  \cite{Yang2003, Shneiderman1992, LeBlanc1990},  graphic-based and geometric \cite{Noirhomme-Fraiture2002}. Visualizations of large data sets are called \textit{visual scalable} visualizations. Visual or perceptual scalability is defined as the capability of visualization tools to display large data sets in an effective manner  \cite{Eick2002}. Eick described that an ideal measure of visual scalability would be the \textit{number of insights} caused by a visualization tool\cite{Eick2002}. The definition of insights differ from 'Aha' moments to knowledge building. While 'Aha' moments can be measured by neural activity knowledge building is harder to grasp. As Information Visualization uses the term insights in both ways Eick defined an initial step to measure the visual scalability by measuring the scalability of visualizations \textit{(Visualization Characteristics}) as well as the scalability of tools \textit{(Database Metrics)}. 
Another important factor of large data visualization is interaction. Interactions with large databases are covered in  \cite{Buono2005, Jerding1998, Mackinlay1991, Keim2005}. Keim et al.  \cite{Keim2005} cover general interaction techniques while Mackinlay et al. introduce new interaction techniques for large data  \cite{Mackinlay1991}. Buono2005 et al. describe the system \textit{TimeSearcher} and the implemented interaction techniques  \cite{Buono2005}. Visualization, data reduction and interaction techniques are 
\par
% Topic: Survey of visualization tools
In the last chapter we are comparing different visualization tools and their ability to display large data.  Bikakis et al. surveyed different generic visualization systems and graph-based systems in the semantic web  \cite{Bikakis2016}. They compared the spectrum of analytical methods and visualization techniques. Yet,  these systems are usually not used in business. 
Other tool comparisons are published by Zhang et al.  \cite{Zhang2012} and Patil  \cite{Patil}. These works compared commercial visual analytics system in the era of Big Data. In contrast Harger et al. surveyed open source visual analytics systems  \cite{Harger2012}. For BI and Analytics the technology research center Gartner publishes a market overview of the most important Business Intelligence(BI) and Analytics tools which also serve for data discovery every year. The leading tools also appear in the survey  \cite{Evelson2012} regarding commercial advanced visualization tools and the works of Zhang  \cite{Zhang2012}. All the surveys consider visualization,  interaction and analysis capabilities.  
\par
This work will contribute to the visualization of large data by focusing on time-oriented data. We will study different visualization techniques for time-oriented data in business and point out requirements for visualizing large time-oriented data in visualization systems. In a next step,  visualization tools which are used in business,  are selected and the requirements are checked. Finally,  we give a recommendation for the tool use depending on the ease-of-use and programming skills and show further research topics.








\chapter{Business Information Visualization of time-oriented Data}
\label{chap:BIV}


\iffalse
\listoftodos

\section{Outline} \todo{Remove from BA}
\begin{enumerate}
    \item The role of InfoVis in Business. Visual Analytics. Selfservice. Insights in Company/ Business. $\Rightarrow$ Business Data
    \subitem But: Other data types also in Business: IoT. Not covered because it is a too wide topic.
    \subitem Business use VA tools to get insight into data.
    \item Need for good visualizations for Business data
    \subitem What are business data? $\Rightarrow$ data types
    \item 2nd challenge: BigData.
    \subitem Where does BigData occur? Application Areas. Streaming Data
    \subitem Is BigData relevant for Business Data? Application Areas of Business Data
    \item Solutions in Literature
    \subitem Aggregation
    \subitem Abstraction
    \item Tools in Business
    \subitem Requirements: Visual Analytics Tools
    \subitem New Visualization Techniques -> Extensionability (p.12, "open
framework fed with pluggable visual and analytical components for analyzing
time-oriented data is useful. Such a framework will be able to support multiple
analysis tasks and data characteristics, which is a goal of Visual Analytics."
\cite{Aigner2007})
    \subitem Market Relevance: Qlik, Tableau
    \subitem Other Approaches: Jaspersoft(because its scalable),
    \subitem Comparison
    \item Conclusion: Which tool for time-oriented data?
\end{enumerate}
\fi
----- \\*
% Business Information Visualization
\section{The Role of Visualization in Business} \label{BIV}
Tools for Business Intelligence (BI) and Analytics, visualization, data discovery as well as for data mining continually gain importance in companies to assist humans to gain insights into their data. Although the differentiation between those tools is not selective and the terms are not clearly distinguished we can make the following differentiation: Data Mining tools allow automatic decision-making by applying algorithms to the data and extract patterns in an automatic way\cite{Goebel1999} while exploratory data analysis (EDA) tools are used to mine data with support of human input. We will use the definition of EDA tools if we speak of visualization tools in this work. As a pwc-survey showed eventhough automatic ways for decision support exists data analysis stil relies on human judgement and thus\cite{PwC2016}, visualization tools are used to support the business user in the data discovery process.

Decision-maker usually are part of the management domain and thus, come from the marketing, sales or management-field but the majority of them uses these tools with few computer science background. Therefore, tools have to be self-explaining, easy-to-use \cite{Crapo2000} and without the requirement of extensive programming. Especially visualization plays an important role as it reduces the information overload\cite{Keima} and simplifys the process of problem-solving\cite{Zhang}. Eventhough we only consider visualization tools which are used to explore data visualization tools have the two roles of presentation and exploration\cite{Crapo2000}. Visualization as presentation is either used to display data without any data mining algorithm or visualization as presentation is used to present the results of a data mining algorithm. Visualization as exploration is used before and during the data mining algorithm to explore the data interactively. This group is called visual analytics. The decision-maker needs both processes for decision making as results are presented on the screen and to explore the data interactively\cite{Ware2012a}. 
Talking of visualization an important data type for business is time-oriented data(\ref{data}) as it allows business to analyze the past and predict the future of the company\cite{Ao2010}. We will have a closer look at user tasks in section \ref{tasks}.
 
\begin{figure}[H]
    \centering
        \scalebox{.5}{\includegraphics{src/images/VisPipeline}}
    \caption{Visualization Pipeline \cite{Ware2012a}}
    \label{fig:my_label}
\end{figure}

\section{Large Data}
Nowadays, the challenge in BIV is to handle large amounts of data and displaying them in an effective manner. The effective manner was defined by Shneiderman \cite{paterno1997concurtasktrees, Shneiderman2008, Keim2008}: \textit{"Overview first, zoom in and filter, then details on demand}. Overview is a basic yet important task because it navigates the user in the data and allows further analysis. But as large data appears with a number of challenges creating this overview is more difficult:
\\*
\textit{Overlap}: 
If the whole dataset is visualized or even a subset the data may overlap and reduce visual clutter.
\\*
\textit{Visual Noise}: 
Even if data items are not overlapping data in large datasets might be to similar to each other. Thus, the user cannot differentiate distinct items on the screen.
\\*
\textit{Limited monitor resolution}:
Even if large monitors are used to visualize data in the end the available pixels on the monitor is smaller than the number of data items in a dataset. 
\\*
\textit{Limited visual perception}:
Moreover, human perception is limited.
\\*
\textit{Information Loss}:
While reducing the number of data items to present them on the limited screen informations in the data can be lost. The important question is which data characteristics to keep such that the user tasks still can be supported.

This is why Shneiderman itself extended its own mantra with the use of aggregation markers and Keim formulated it as \textit{Analyze First - Show the Important - Zoom and Filter, and Analyze Further - Details on Demand}\cite{Keima}. The effective representation of large amounts of data requires to extend the Overview by the use of aggregation which can be achieved by appropriate techniques, correct parameterization, interaction and analytical methods\cite{Aigner2008}. 


% Users in BIV are usually no computer scientist. [Share of Workers in Business] 
% Self-Service Data
% Role of Visual driven analysis -> need for tools





\section{Definitions}
In order to give a clear impression of the used terms we define frequently used expressions.

\textbf{Information Visualization}\\*
Information Visualization is the graphical representation of non-spatial or abstract data\cite{Keim}. In contrast to scientific visualization the data which is visualized in information visualization does not have an inherent 2D or 3D structure\cite{Shneiderman2008}. 
As business data usually is abstract, discrete and multi-variate according to \cite{Tegarden1999} the type of visualization for business data is Information Visualization.

\textbf{Visualization technique}\\*
The way how data variables are mapped to graphical primitives is called visualization technique. Typical examples are bar charts, line charts or scatterplots. Thereby, every technique has its own philosophy how to present the data (called \textit{Visual Metaphor}\cite{Tegarden1999}), its own strengths and weaknesses and its particular application. Typically new visualization techniques are created for a specific application and the available amount of visualization technique is huge and continually growing. In this work we will focus on \textit{time-oriented data} and thus, only consider visualization techniques for this cause. Moreover, we will exclude visualization systems which are combined techniques. 
\\*

\textbf{Visual Data Exploration}\\*

\textbf{Time-oriented Data}\\*
Time-oriented data is data which is linked to time\cite{Aigner2011}. Often, time-oriented datasets are very large and multi-variate which makes it difficult to analyze them. The question is how time-oriented data can be analyzed if the number of data points exceeds the screen resolution. This brings us to the definition of large time-oriented Data. 
We define large time-oriented Data as abstract time-dependent data which is too large to fit on the screen. \cite{Shneiderman2008} Multi-variate time series are time series where one data item holds severable variables at the same point of time \cite{Aigner2011}.
In the following work, we will use time-oriented Data, time-oriented Big Data and Big time-oriented Data equivalently for large time-oriented Data. 




\textbf{Visualization Tools}\\*
While BI is defined as...\todo{Definition BI} data mining describes the extraction of patterns and models of the underlying data structure\cite{FerreiradeOliveira2003}. When data mining is used together with visualization data mining is based on automated algorithms which detect relevant patterns and display the results afterwards. In contrast, visual data exploration is a completely human guided process\cite{FerreiradeOliveira2003}. First data is displayed on the screen as a visualization and with the help of human visual capabilities new hypothesis are formed. Data visualization is a more general term for generating a graphical representation out of data and is used as well in BI, Analytics, data discovery as in visual analytics. 
Visualization tools display hundreds of items on the screen and offer interaction techniques such as zooming and filtering\cite{Shneiderman2008}.


% Data types for Information Visualization
\section{Data type} \label{data}

For choosing appropriate visualization techniques the first step is in understanding the underlying data and creating a correct data model\cite{Aigner2011}, so that the visualization technique represents the underlying data structure and offers best insights\cite{Bacic}. \textit{Aigner et. al.} proposed  the following questions to model the visualization problem: 

\begin{itemize}
    \item What is presented?
    \item Why is it presented?
    \item How it is presented?
\end{itemize}
\textbf{What is presented?}\\*
Talking of large time-oriented data for business we will consider the three given characteristics of the data: characteristics of business data, time-dependency and their size. Business data is collected in many different areas. The following table gives an overview about the applications: 

\begin{table}[th]
	\centering
	% caption format: \caption[<business aplications>]{<long version>}
	\caption[Table 1]{Business applications\cite{Brachman1996,Tegarden1999}}
	\label{businessapplications}
	\begin{tabu}{ll}
	\toprule
	Marketing & Financial Sector \\
	Fraud Detection & Manufacturing and Production \\
	Operations Planning & Market Analysis \\
	Health Care & Network Management\\
	\bottomrule
	\end{tabu}
\end{table}

For each application area exists a number of publications covering different aspects. Thus, the analysis of every single application would go beyond our scope and we decided to consider data with the following characteristics: 
\begin{enumerate}
    \item The data is structured. 
    \item The data is abstract.
    \item The data is multi-variate.
    \item The data is discrete.
\end{enumerate}
\textbf{Structured data}: Data can come in many different forms. Unstructured data appears in text, speech and language processing\cite{Borgo2013}. Structured data comes in tables in which each attribute are represented by one column and each row is one data item. The attributes can be either numerical or text-based. As multi-variate data is usually presented in tables\cite{Borgo2013} we assume that business data is structured.\\*
\textbf{Abstract data}: Abstract data is defined as data without any spatial relationship in the data\cite{Shneiderman1996}. \\*
\textbf{Multi-variate data}: 
Often, multi-variate is mixed-up with the term multi-dimensional. For this reason we define  multi-variate data by the number of dependent attributes. If the dataset holds more than two dependent attributes then we call the data \textit{multi-variate}. In contrast, multi-dimensional depicts the number of independent attributes of a dataset\cite{Aigner2011}.  \\*
\textbf{Discrete and time-oriented data}: Talking of time-oriented business data we assume that the each data item is linked to time. Thus, one table column represents time. The time-dependency of the data structures the data by a given order. Every data item is mapped to a specific point in time with a smallest possible unit such as seconds. Time with a smallest unit is mapped to integer\cite{Aigner2011} and thus we assume that time-oriented business data is discrete and has a given order. Ordered discrete data is called \textit{time series}. The time series is an ordered sequence of n data items $T=(t_1+t_2+...+t_n),t_i\in\mathbb{R}$. Thus, we will focus on the analysis of time series data. \\*
These assumptions are based on the work of Tegarden in which business data is described as abstract and multi-variate and discrete\cite{Tegarden1999}.
\\*
\todo{drin lassen? Raus?}
Eventhough, data can be point-based or interval-based (scope), linear or cyclic (arrangement), ordered branching and with multiple perspectives(viewpoint)\cite{Aigner2011}. We will not narrow our focus further but talk on an abstract level. Our perpective is to explore visualization techniques with different scopes, arrangements and viewpoints according to their scalability but of course, every visualization technique is designed for a specific scope such as linearity or seasonal behaviour. The decision for a specific visualization technique is still up to the user. \todo{hier evtl. die unterscheidung von radial und nicht radial einführen / linear oder zyklisch?}.\\*
\textbf{Large data:} Lastely, we characterize the size of the data. Therefore, we stick to a definition of Huber. He devided data in small, medium, large, huge and massive data. Large data according to\cite{Huber1994} is defined as datasets with $10^6$ and huge data with $10^8$ data entries. We are considering large and huge amounts of data. Nowadays, companies strive to do \textit{Big Data}. Big Data in typically defined as data with  high volume, high velocity, high veracity and high variety\cite{Wang2015}. The contribution of our work to Big Data is the study of presentation of of high volume datasets.



% User Tasks
\section{Time-oriented User Tasks} \label{tasks}
\textbf{Why is it presented?}\\*

Every tool for decision-support should consider the user perspective. The business user is a kind of person which is interested in verify existing hypothesis (verification) and discover new patterns (discovery) and by using the tool he expects the tool to assist him in analyzing the data, finding critical points and perform analysis automatically\cite{Brachman1996}. Verification and discovery in the analysis of time-oriented data can be split in seven major tasks\cite{Esling2012}:

\\*
\textbf{T1: Query by Content}
\\*
Query by content describes the retrieval of similar items to the query and returns a set with the most similar solutions to the query. In time-oriented data query by content returns the \textit{k} most similar time series to the queried time series.
\\*
\textbf{T2: Clustering}\\*
Clustering is the process of finding expressive groups (clusters) out of the data. Therefore, the dataset is devided into subgroups according to some similarity measure. In the context of large time-oriented data clustering is important to compare similar time-series.
\\*
\textbf{T3: Classification}\\*
In classification the task is to find the right group the item belongs to. According to \textit{Aigner et al.} temporal classification describes the preprocess of finding the correct group for given data or dataset. This task is important for large data to abstract the data and make them handable. As this task is preprocessing and not data presentation or exploration we will not check whether the tools support the user in this task. 
\\*
\textbf{T4: Segmentation}\\*
Segmentation splits a time series into \textit{k} meaningful subsequences (segments)\cite{batyrshin2007perception}. 
\\*
\textbf{T5: Prediction}\\*
In Prediction\textit{k} future events are predicted based on the past \textit{n} time series. This process is also known as \textit{forecasting}. Forecasting 
\\*
\textbf{T6: Anomaly Detection} \\*
Anomaly detection points out events which behave in a different way than expected.
\\*
\textbf{T7: Pattern Discovery} \}\*
Pattern discovery finds regularly appearing structures in a time series.  It covers the exploration of trends, outliers and clusters. Especially, in business this task is one the most important tasks.\todo{quote finden}


\iffalse
\begin{tikzpicture}[sibling distance=12em,
  every node/.style = {shape=rectangle, 
    draw, align=center,
    top color=white}]]
  \node [shape = ellipse] {Visualization Tasks}
    child { node [shape = ellipse] {visualization methods} %time-oriented data: 
      child { node [shape = ellipse] {right visualization method} }
      child { node [shape = ellipse] {right parametrization} 
        child { node {navigation in time} }
        child { node {search} }
        child { node {comparison} }
        child { node {manipulation} } } }
    child { node [shape = ellipse] {analytical methods} %large time-oriented data: 
      child { node {aligned at}
        child { node {relation sign} }
        child { node {several places} }
        child { node {center} } }
      child { node {first left,\\centered,\\last right} } };
\end{tikzpicture}
\fi

% Visualization Techniques
\section{Visualization of large-scale time-oriented Data} \label{vis}
\textbf{How is it presented?}
% introduction of methodology: selection of vizTechniques, introduction of classes
The following section discusses several visualization techniques for time-oriented data including the question of their scalability because the key to a successful representation of the data is the choice of an appropriate technique. This is shown in the cognitive fit theory\cite{Vessey1991} as this theory showed a significant increase of user task performance the better the visual representation of a problem fitted to the cognitive user model. Visualizations of time-oriented data therefore have to represent the cognitive user model. Based on the findings in literature we assume that the cognitive user model is represented by the user tasks which we described in section \ref{tasks}.\\* 
\textbf{Standard Visualizations of time series data}
Usually time series data is visualized with line charts. However, line charts can only show \textit{univariate} data. Our selection of visualization techniques is based on Aigner et al.\cite{Aigner2011} which presents current approaches to visualize time-oriented data. Hereby, we focus on 34 techniques which can be used to show business data: abstract, multi-variate and discrete data (compare section \ref{data}). 
As the work of Aigner was published in 2011 we completed the set of techniques with current approaches based on the \href{http://survey.timeviz.net/}{TimeVizBrowser}. Moreover, we consider only stand-alone visualization techniques which are techniques no systems, tools or software. In literature a significant amount of publications describe new tools or systems which tackle the visualization of time-oriented data. These tools usually are specific tools which can only be applied in a limited field. As we write this work with the perspective of business users tools have to be generic and single-task systems are not appropriate.

We are aware that this discussion cannot be exhaustive as time-oriented data is a current research area and day-to-day new visualization techniques are developed.
Moreover, time-oriented data appears in different areas of business: E-commerce, Smart Health, E-Government, Science \& Technology, Security \& Public safety. Each sector collects different types of data and uses different applications, which makes it impossible to name every single existing visualization technique.


However, we think that this work gives a good overview of visualization techniques as we stick to Keim's taxonomy\cite{Keim1995} of visualization techniques for multi-variate data which classifies them into the classes: \textit{geometric-projective}, \textit{graph-based}, \textit{hierarchical}, \textit{icon-based} and \textit{pixel-oriented}.
\\*
\textbf{Graph-based} techniques present large graphs by using layout algorithms\cite{Keim1996}.
\\*
\textbf{Geometric projection} techniques (GP-techniques) map multi-dimensional data to the 2D screen\cite{FerreiradeOliveira2003}.
\\*
\textbf{Pixel-oriented} techniques map each data item to one pixel on the screen. Position and color are used to represent data attributes\cite{Keim1996}.
\\*
\textbf{Hierarchical} techniques divide the k-dimensional space into subspaces and shows them hierarchically. 
\\*
\textbf{Icon-based} techniques map each data item onto one icon. The attributes are mapped to different icon features\cite{Keim2001}.

\section{Visual Scalability}

This work studies how well these techniques scale to large data. To define the ability of visualizations to present large amounts of data we introduce the term \textit{visual scalability}.
Visual or perceptual scalability is defined as the capability of visualization tools in displaying large datasets in an effective manner\cite{Eick2002}. In the context of time-oriented business data effective means the presentation of patterns to support the user tasks. To measure the visual scalability of different visualization techniques for time-oriented data we refer to the work of Eick\cite{Eick2002}. He proposed to measure visual scalability by the database metrics of the dataset and the visual characteristics of the visualization technique. \\*

\textbf{Database metrics} measures the \textit{size of the database} in bytes, the number of rows or the number of attributes at the level of the visualization tool. For multi-dimensional data the \textit{database metrics} are a combination of the number of rows and the attributes. \\*

\textbf{Visualization characteristics} describe the number of elements and attributes presented on the screen, thus measuring how many distinct items a visualization technique can display. This number is measured on the visualization technique level.
\\*
The combination of the database metrics and the visualization characteristics describes the scalability of a visualization tool. Furthermore, the tools scalability is influenced by more factors which will be discussed later (\ref{factors}).

\subsection{Visual Scalability of time-oriented techniques}
In the analysis of the techniques we discovered that time-oriented techniques are covered by only four classes. Thus, we measured the scalability of four classes. 
 Based on this analysis we identified a number of \todo{Anzahl der Techniken einfügen} techniques which are appropriate to show large time-oriented data. These techniques will be described in detail.

\begin{table}[H]
	\centering
	% caption format: \caption[<Scalability of Visualization Classes>]{<long version>}
	\caption{Scalability of Visualization Classes}
	\label{vizScalability}
	\begin{tabu}{  | l | l | l |}
	\toprule
	Visualization Class & Technique & References\\
	\midrule
	    \multirow{10}*{Geometric} 
		& EventRiver        & \cite{Luo2012}\\
		& Flocking Boids    & \cite{Moere2004}\\
		& Intrusion Detection\\
	    & Kiviat Tube       & \cite{Tominski2005}\\
        & Layer Area Graph\\
        & MultiComb         & \cite{Tominski}\\
        & Multi-resolution CircleView & \cite{Keim2005}\\
        & Parallel Glyphs   & \cite{Fanea2005}\\
        & Temporal Star     & \cite{Noirhomme-Fraiture2002}\\
        & TimeWheel         & \cite{Tominski}\\
        & Worm Plots\\ \hline
        \multicolumn{3}{|p{\linewidth}|}{
        \textbf{Geometric projection} techniques (GP-techniques) map multi-dimensional data to the 2D screen\cite{FerreiradeOliveira2003}. The mapping-function often includes data reduction techniques (see section \ref{analytical}). Thus, the visualization characteristics of GP-techniques strongly depend on the mapping-function. When data reduction techniques are involved, this class of techniques becomes a high-potential class for large datasets as they allow to present large to huge data.} \\ \hline
        
		\multirow{3}*{Hierarchical} 
		& Pixel-Oriented Network Visualization \\
		& Software Evolution Analysis & \cite{}\\
		& Timeline Trees \\ \hline
		\multicolumn{3}{|p{\linewidth}|}{\textbf{Hierarchical} techniques: we analyzed three hierarchical visualization techniques. One technique included aggregation techniques by collapsing the nodes. With collapsed nodes the technique can display large to huge amount of data. Pixel-oriented Networks use clustering and thus, can.} \\ \hline
        \multirow{5}*{Icon-based}
        & Gravi++\\
        & InfoBUG\\
        & PeopleGarden\\
        & Spiral Graph\\
        & VIE-VISU\\ \hline
        \multicolumn{3}{|p{\linewidth}|}{
        \textbf{Icon-based} techniques: As every data item requires one icon icon-based techniques can show less data items than the number of pixels on the screen. When showing large data icon-based techniques face the challenge of clutter and occlusion\cite{Borgo2013}. Thus, icon-based techniques can display small- to medium-sized datasets.}\\ \hline
        \multirow{12}*{Pixel-oriented}
        & 3D ThemeRiver & \cite{Imrich2002}\\
        & Braided Graph\\
        & CircleView\\
        & Data Tube Technique\\
        & history flow\\
        & Kaleidomaps\\
        & Pixel-Oriented Network Visualization\\
        & Recursive Pattern\\
        & Spiral Display\\
        & Stacked Graphs\\
        & ThemeRiver\\
        & Time Curves\\
        & TimeRider\\ \hline
        \multicolumn{3}{|p{\linewidth}|}{
        \textbf{Pixel-oriented} techniques: Since only one pixel per data item is used this class can maximize the used screen space. Let $M$ be the monitor resolution with the screen-width $w$ and the screen-height $h$, $P$ the number of pixels in $M$ and $D$ the maximum of data which can be displayed at once. In pixel-oriented techniques  \begin{math}
        D = w*h
        \end{math}
        which shows that pixel-oriented techniques can display large, but not huge data.} \\ \hline
	\bottomrule
	\end{tabu}
\end{table}


Most of the existing visualization techniques nowadays still are not appropriate in visualizing large data. Pixel-based visualizations represent each data item by one pixel and thus, are limited to 2 mio. pixels. Even new visualization techniques which where developed to visualize "very large" datasets follow the pixel-oriented approach\cite{Keim1995, Keim1996}. Icon-based visualizations display one data item per icon and thus can display even less data than pixel-oriented visualizaton techniques. The most promising techniques are hierarchical and geometric-projective techniques as they combine aggregation or abstraction with visualization. Their scalability depends on the mapping-function. 
One interesting extension of pixel-oriented techniques is the \textit{multi-resolution} approach\cite{Keim2005}. The idea is to show more relevant data items at a pixel-based level and less relevant data items in an aggregated way. In Multiresolution CircleView recent data is shown at full resolution in the middle of the circle and previous-year-data is placed at the outer circle. With multi-resolution it is possible to extent the pixel-limit from $w*h$ to larger datasets. 

\begin{table}[H]
	\centering
	% caption format: \caption[<Scalability of Visualization Classes>]{<long version>}
	\caption{Scalability of Visualization Classes}
	\label{vizScalability}
	\begin{tabu}{ l | c }
	\toprule
	Visualization Class & Scalability\\
	\midrule
	Geometric &  \cellcolor{green!25 } > 2 mio. pixel\\
	Hierarchical & \cellcolor{green!25} > 2 mio. pixel \\
	Icon-based & \cellcolor{red!25} $\leq$ 1 mio. pixel \\
	Pixel-oriented & \cellcolor{yellow!25} $\leq$ 2 mio. pixel \\	
	\bottomrule
	\end{tabu}
\end{table}



\iffalse
 The discussion whether a visualization technique is part of the standard visualization or belongs to advanced visualization is not unambigiously. \textit{Aigner et. al} classify Parallel Coordinates as a standard visualizations\cite{Aigner2011} whereas \textit{Keim et. al.} \cite{Keim} are talking about Parallel Coordinates as a novel techniques. This discussion of course is determined by the time epoche. The longer a visualization technique is available the more it is counted as standard visualization technique. But defining a time-period after which the visualization technique is seen as standard is not possible as other factors influence the judgment of standard or advanced data visualization, such as for example the degree of familiarity. Nevertheless, researcher tried to define advanced data visualization. Russom stated "Advanced Data Visualization (ADV) is able to "scale the visualizations to thousands or millions of data points, can handle different data types and present analytical data structures." \cite{Russom2011}. Thus, we define advanced data visualization as large-scale data visualization techniques which are able to scale to large and huge amounts of data. 
 \\*

%Aspects of Scalability:
% Wie viele Datenpunkte sind notwendig, um Pattern darzustellen? -> data 
% Interaction Techniques
% Downsampling -> Analytical Methods

 The challenges of large-scale data for ADV are \textit{scalability} and \textit{dynamics}\cite{Wang2015}. With its volume the challenges for large-scale data are also challenges for Big Data defined as high volume, high velocity, high veracity and high variety datasets\cite{Wang2015}. In this work we concentrate on the scalability challenge for visualization techniques. The challenge is in finding appropriate techniques\cite{Aigner2008,Keim2005} which scale to large amount of data.
 
\fi

\subsection{Visualization Techniques in Detail}
\subsubsection{Geometric-Projection Techniques}
Geometric-projective visualizations seems to be the most scalable visualization technique depending on the mapping function. For this reason we will explore techniques in the geometric-projective class in detail and discuss their scalability.
Furthermore, we suggest to devide GP-techniques into \textit{radial} and \textit{non-radial} visualizations\cite{Diehl2010} as our analysis has shown that a large part of GP-techniques is based on a radial layout. As radial GP-techniques share common properties such as the number of attributes this differentiation is made to analyze the visual scalability of GP-techniques.\todo{definition of radial wie in \cite{Diehl2010}?} \\*


\begin{table}[H]
	\centering
	% caption format: \caption[<Radial and non-radial GP-techniques>]{<long version>}
	\caption[Table 1]{Radial and non-radial GP-techniques}
	\label{radialTable}
	\begin{tabu}{lcc}
	\toprule
	GP-Technique & radial & non-radial \\
	\midrule
	EventRiver &  & x \\
	Flocking Boids &  & x \\
	Kiviat Tube & x &  \\
	MultiComb & x &  \\
	Multi-resolution CircleView & x &  \\
	Parallel Glyphs & x &  \\
    Temporal Star & x &  \\
	TimeWheel & x & \\
	\bottomrule
	\end{tabu}
\end{table}

Radial layout techniques can scale up to 10-20 attributes.

\textbf{EventRiver} was created in journalism to compare hot topics and their relevance over time. This technique uses clustering algorithms to analyze frequent words. Colored Bubbles with different sizes are placed along the x axis according to time. The bubble size represents one cluster and its relevance. The shape shows when the topic appeared and disappeared. Color and the position on the y axis are used to group topics together.
Due to the clustering analysis before the rendering of the visualization data is grouped and thus large data can be displayed. 
EventRiver comes along with interaction techniques such as filtering, reordering and zooming.
\\*
\textbf{Flocking Boids} simulate the behaviour of data items in 3D. Thus, data items are represented by a colored, curved line with changing transparency. Based on boid simulation behaviour based rules define the position of the data item over time and its velocity. Different variables can be compared by creating several flocking boids next to each other. Flocking Boid was tested with 12.631 data entries\cite{Moere2004}. 
Analytical Methods such as clustering or subset selection are outsourced to database algorithms and interaction techniques are not implemented but could be extended\cite{Moere2004}.
\\*
\textbf{Kiviat Tube} is a unfolded Radar Chart along the z axis in 3D. Several Radar Charts are stacked behind each other along the time (z) axis and form a tube. Thus, variables are mapped on radial aligned planes and can be compared. Interaction such as changing the planes positions and navigating through time enables the user to compare different variables over time.\\*
The number of attributes is limited to approximately 10-20 attributes as the radial layout limits the number of variables. Experiments which measured the maximum number of variables doesn't exist. 
\\*
In \textbf{MultiComb} \textit{k} time series plots are mapped on a circle in two possible ways. One way is to position the plots along the circumference. Or the plots are mapped perpendicular to the circumference. In this option the plot resembles a star.  
\textbf{Multi-Resolution CircleView} extends the CircleView technique by aggregating data according to their relevance. Similar to CircleView the circle is devided in k segments and k is the number of attributes. The least relevent data is placed at the outer circle with a high aggregation level and the most relevant data in the inner circle. The higher the relevance the lower the aggregation level. The number of displayed data items thus depends on the relevance function. 
\\*
\textbf{Multi-resolution CircleView} enhances the CircleView technique. Instead of mapping each value to one pixel, data items are aggregated. More recent items are placed in the middle of the circle. These items are represented by one pixel each. Less relevant items are aggregated and placed at the outer part of the circle. These items are usually from a preceding point of time.
\\*
\textbf{Parallel Glyphs} pair Parallel Coordinates with Star Glyphs. While similar to Parallel Coordinates each data item is represented by a polyline which connects the vertical axis (attributes) the attribute axis are radially unfolded in 3D and show the data value of the data item over time. Thus, each data value over time is represented by a star glyph. The visualization can be expanded by connection lines over star glyphs. Through the extension of 2D to 3D parallel glyphs are able to display more data rows than parallel coordinates (PC). PC had the problem of clutter while displaying 15.000 data on a gray-scale.  items\cite{Keimb}.
Parallel Glyphs provide brushing of polylines, filtering, axis reordering, rotating in 3 directions, transparency support if the glyphs overlap each other, focus+context presentation through magnification lenses.
\\*
\textbf{Temporal Star} aligns multiple attributes in a star-like manner around the centre. Each star is one point of time. The time axis connects several stars to a 3D-object.
\\*
\textbf{TimeWheel}is a 2D technique. Similar to variation one of \textit{MultiComb} attribute axis are positioned along the circle circumference. In the centre of the circle the time axis is placed. 
\\*

In conclusion a visualization tools should offer the possibility to create multi-resolution and geometric-projection techniques. 

\section{Analytical Methods}\label{analytical}
Comparing every visualization technique the need for data reduction becomes obvious. In the literature \textit{data abstraction and aggregation} are well know techniques for data reduction\cite{FerreiradeOliveira2003,Aigner2011, Keim2005}. There exist two ways to data reduction: reduce data horizontally or vertically. 
Vertical data reduction describes the process to remove data rows whereas horizontal data reduction is used for dimensionality reduction. 
\begin{figure}[H]
    \centering
        \scalebox{.1}{\includegraphics{src/images/dimreduce}}
    \caption{Horizontal Data Reduction}
    \label{fig:my_label}
\end{figure}

\begin{figure}[H]
    \centering
        \scalebox{.1}{\includegraphics{src/images/aggregation}}
    \caption{Vertical Data Reduction}
    \label{fig:my_label}
\end{figure}

\subsection{Vertical Data Reduction}
One way to decrease the size of large or huge data sets is to remove data rows. This section lists several data removal techniques. One important issue for every technique is the question which data to keep and which data to remove. The disadvantage of data reduction is the information loss.
\subsubsection*{Sampling}
Sampling describes a strategy to reduce data by creating a sample of the original data. Thereby, sampling is scalable, reduces clutter, preserves information of the kept data as well as patterns and trends\cite{PiringerHarald2011}. Still, sampling may eliminate outliers or single data items and does not provide any garantuee to avoid visual overlap. 
\subsubsection*{Filtering}
Filtering is a method to reduce data by some specific criteria. In visualization filtering often is based on user input such as dynamic query sliders. Thereby, filtering can support the user in excluding task-irrelevant data portions and unlike sampling filtering may be appropriate to detect outliers. However, as filtering is based on the exlusion of unrelevant attributes it does not garantuee a specific target size of the data set. In some cases the target size might still be too large. Moreover, filtering also does not garantuee to discriminate distinct data items\cite{PiringerHarald2011}.
\subsubsection*{Temporal Data Abstraction}
Temporal Data Abstraction\cite{Aigner2011} reduces the number of data rows by focusing on relevant concepts, patterns, shapes over time and neglecting irrelevant details. Clusters and summery statistics\cite{PiringerHarald2011} are typical examples for data abstraction. However, the authors found a trade-off between abstraction and accuracy: with low abstraction and a high accuracy there exists the problem of cluttering. \todo{besser integrieren}
One way to implement temporal data abstraction is to use natural language processing in visualization tools. \todo{wie formuliere ich das am besten? Answerrocket + google}
Clustering as defined in the user tasks~\ref{sec:user} have the advantage of reducing visual clutter by displaying the natural groups of the data instead of every single data item. Clustering has the advantage of pattern and outlier preservation if the similarity measure is appropriate.
\subsubsection*{Aggregation}
Aggregation describes the process of grouping several data items together. Hierarchical aggregation builds aggregated data items by forming a tree structure and collapsing the children of a tree\cite{elmqvist2010hierarchical}. Binned aggregation devides data into adjacent bins and combines them for aggregation\cite{Liu2013}. Pixel-aware aggregation clusters pixels according to their screen coordinates\cite{li2016polyspector}. M4 aggregation compresses time series data into a set of equidistant time spans\cite{jugel2014m4}.
As time-oriented data has specific characteristics analytical methods have to consider these peculiarities. One way to reduce data size with respect to the time-specific characteristics is temporal aggregation. Hereby, data is aggregated according to the time unit (day, month, year) in temporal hierarchy levels. Examples for temporal aggregation are hierarchical axis \cite{Chung2014} which enable to navigate in time. \todo{mich damit beschäftigen}
\\*
Other ways to reduce data vertically are binning and pivotization. We will not explain them here in detail as they treat continuous and hierarchical datasets which are beyond our scope in this work. 

\subsection{Horizontal Data Reduction}
Besides data removal datasize can be reduced by decreasing data dimensionality. Since business data often is multi-dimensional but visualization techniques are limited in the number of attributes dimensionality reduction is a way to process data sets in a way that they can be displayed by visualization techniques. 
A common way to map a high-dimensional to a low dimensional space is the Principal Component Analysis (PCA)\cite{Aigner2008}. Other approaches are Multi-Dimensional Scaling or Self-Organizing Maps\cite{PiringerHarald2011}. The advantage is that they keep the distance between two points after the projection. However, the attributes in the low-dimensional space are difficult to understand which makes them unintuitive. To overcome this problem hierarchical dimension reduction has been proposed. \todo{Quellen einfügen + evtl. ausformulieren}
%Segmentation Techniques, Factor Analysis, Multidimensional Scaling, FastMap\cite{FerreiradeOliveira2003} Correlation analysis, Information gain or Statistical methods, Sampling, Clustering or Aggregation\cite{Keim2005}



% 3 Methods for Large Data (Aigner2008)
% 1. Temporal Data Abstraction: VIE-VENT + The Spread
% 2. Data Abstraction for Multivariate Data: PCA
% 3. Data Aggregation



\subsection{More factors} \label{factors}
Moreover, besides the database metrics and the visualization characteristic visual scalability is influenced by six factors: 
\begin{itemize}
    \item Human Perception\cite{Keim2005,Deering1998}
    \item Monitor Resolution 
    \item Visual Metaphors
    \item Interactivity
    \item Data structures and algorithms
    \item Computational infrastructure
\end{itemize}
\todo{Entscheiden, inwiefern ich diese 6 Faktoren mit aufnehme. Wenn ich sie mit aufnehme, muss ich sie auch erläutern}

\subsubsection*{Human Perception} 
% What can humans perceive? How many pixels? 
One way to measure the viual ability to see details is called \textit{visual acuity}\cite{Ware2012a}. 
According to experiments discussing the limits of human perception \cite{Deering1998} the human visual system is able to perceive 15mio pixels per eye. Assuming that the amount of perceivable pixels for two eyes is larger than 15mio pixels but smaller than 30mio pixels due to the overlap of the field of view the max. amount of perceivable pixels (pp) is:
\begin{math}
15 mio. \leq pp < 30 mio.
\end{math}
Nevertheless, the important question is not the amount of perceivable pixels but whether the data structure, patterns, trends and further information in the data can be perceived. To achieve this goal Keim\cite{Keim2005} created \textit{CircleView}.
%Talking of scalability visualization techniques should be able to scale according to their data types, data sources and levels of quality \cite{Keim2008}. 
\subsubsection*{Focal Depth-of-Focus Information}
\\*
\subsubsection*{Monitor Resolution}
Even though large wall-sized screens have been developed in business usually. \todo{schreiben}
\\*
\subsubsection*{Visual Metaphors}
Improved visual metaphors enhance the scalability of visualization techniques\cite{Eick2002}. \todo{Beispiel} One way to improve visual metaphors are multi-resolution metaphors\cite{Keim2005}. These show less details at the \textit{Overview-Level} of a visualization and more details at the \textit{Detail-Level}. \textit{CircleView} is one visualization technique with a multi-resolution metaphor which clusters data items according to their relevance and displays the clusters first.
\\*

\textbf{Pixel-oriented} techniques map a data point to a colored pixel. Since each data entry requires one pixel on the monitor pixel-based visualizations can maximal display around 2.000.000. data points. Moreover, they use multiple windows and center the most relevant data in the middle and the less relevant data outside the center\cite{Keim1996}.\\*


\section{Parameterization}

\section{Need for Interaction Techniques}
Interaction Techniques describe how the user can interact with the data. As interaction is a crucial factor for scalable visualization techniques it enhances visual scalability\cite{tegarden1999}. 
Hereby, we assume that interaction is in responsibility of the visualization tools. Keim\cite{2002} devided interaction techniques into two classes of techniques. The first group supports the user in exploring the data \textit{interaction techniques} and the second helps in displaying large databases \textit{distortion techniques}. \\*
\textbf{Interaction techniques} include several techniques such as dynamic projection, filtering, zooming, and linking \& brushing.
The main idea of \textbf{interactive distortion techniques} is the presentation of showing the data at different levels of detail. Examples are hyberbolic and spherical distortion, bifocal displays, perspective wall, graphical fisheye view, hyperbolic visualization and hyperbox\cite{Keim2002}\todo{genauere Zitierung}. 

\subsubsection*{Filtering}

\subsubsection*{Selecting}
Selecting of a single data item or a range of data items.
\subsubsection*{Linking}
Linking describes the connection between multiple views. \textit{Linking} in the context of large data is important to provide overview and detail views. 
\textbf{Brushing \& Linking}\\*
The user can select data items on the screen(Brushing) and the respective items are highlighted in every connected window (Linking). Therefore, lasso, rubber-band or rectangular selection enables the user to select groups of data items\cite{tegarden1999, Aigner2011}. In the context of time-oriented data a typical brushing activity is the selection of an smaller time-span to see more details during this period of time.
\textbf{Dynamic Queries}\\*
Besides Brushing \& Linking, dynamic queries provide a filter-mechanism by multiple widgets, such as sliders or input fields\cite{Hochheiser2004,Shneiderman2008,Aigner2011}. A specific dynamic query are time-boxes. These boxes are rectangular selection areas which are drawn by the user. The tool then only displays values with a similar pattern to the pattern in the time-boxes.
\textbf{Focus + Context} shows a selected region of interest in detail while the rest of the dataset is still shown\cite{Keim2005}. 
\textbf{Drill down}
\textbf{Fish-eye Views}\\*
\textbf{Table Lens}\\*
\textbf{Magic Lense}\\*
\textbf{Perspective walls\cite{Keim2005}}\\*
\textbf{Zoom + Filter}\\*



\subsection{Advanced Interaction Techniques}
\textbf{Navigation Maps}\\*
For navigating through massive datasets 
\textbf{Information Mural}
\textbf{Coarse Presentations}

\section{Layout}
Closely related to interaction techniques is an appropriate layout to enhance visual scalability. Ware\cite{Ware2012} showed that zooming is an easy-to-use-tool for a small amount of items. However, if the user needs to keep three items or more in its visual working memory multiple windows are more effective than zooming. Thus, displaying large time-oriented requires a layout with multiple simultaneous views. Coordinated views are linked views. If one data item is selected and brushed the corresponding characteristic in other windows is also highlighted.
\begin{figure}[H]
    \centering
        \scalebox{.3}{\includegraphics{src/images/zoomVSmultiWindow}}
    \caption{Measured task performance of zooming compared to multiple windows. \cite{Ware2012a}}
    \label{fig:my_label}
\end{figure}
Other techniques are \textit{distortion techniques}\cite{mackinlay1991perspective}: \textit{Bifocal Displays}\cite{Spence1982}, \textit{Perspective Walls}\cite{mackinlay1991perspective}

\chapter{Tools}
\label{Tools}
\section{Scalability in visualization tools}
Besides of the visualization technique itself the visualization tool influences the visual scalability in limiting how many datarows can be fetched in each visualization. Qlik Sense limits the initial fetch to 10.000 but gives the opportunity to fetch more data if needed. 
%Definition which tool meant: Tools can be divided into BI Tools, Analytic Tools, Visualization Tools and Custom Tools\cite{Schnell2014}.
%Difference Data Minining and Visual Analytics: First generate new knowledge and then visualiza vs. visualize and generate new knowledge

\section{Selection of Tools}
\textbf{Requirements}
Advanced Data Visualization in business context requires software that is able to scale visualization in an "effective manner"\cite{Russom2011}. Offering advanced visualization techniques, parameterization, interaction and analytical methods such as data abstraction\cite{Tegarden1999,Aigner2011,Eick2002,Zhanga} are core functions of ADV software. Based on the Magic Quadrant for Business Intelligence and Analytics Platforms\cite{Parenteau2016} Qlik, Tableau and Microsoft are the leading visionaries of BI Vendors. \cite{ITCentralStation} as a crowdsourcing recommendation platform for BI tools ranked Tableau, Qlik, Oracle, Microsoft Power BI and IBM Cognos on the first five places.
\textbf{The Role of APIs}
Commercial software tends to need more time for the development and integration of advanced visualization for large data\cite{Zhanga, Simon2014}. To bridge the gap, vendors started to offer a bunch of APIs to create and integrate visualizations.
% data load for BigData: 
\textbf{Software not included in this work}
As the goal of this work is to compare software tools %alternative: provide a market overview
for Visual Analytics in business and time-oriented data we focus on Business Intelligence and Analytics software. Furthermore, the software needs visualization features, the ability to present time-dependent data. Software with one of the following items is intentionally not considered: 
\begin{enumerate}
    \item Software that only presents one-dimensional data. 
    \item Software that \todo{write reasons why Tools are not considered}
\end{enumerate}

\section{Investigation of Advanced Visualization Tools using a Feature Classification Scheme}
In this section we analyze advanced visualization tools in business with a feature classification scheme. We provide the classification scheme based on the collected success criteria of chapter 3 and apply it to current visualization tools. Even though the selection of the tools is not exhaustive, we believe that the reviewed products represent the state-of-the art and provide the table of different tools which was used for chosing the remaining 5 tools in the appendix.
\subsection{The Classification Scheme}
The tools basis of assessment is the classification scheme which is devided into 3 \todo{maybe 4: Layout?} subsections:\textit{Analytical Techniques, Visualization Techniques, Interaction Techniques}. 
Each section contains success criteria which are necessary to display large time-oriented data. To rank the tools in the respecting category we developed the following success criteria score(SCS):

\begin{table}[th]
	\centering
	\caption[criteria]{Succes Criteria Score}
	\label{Succes Criteria Score}
	\begin{tabu}{cl}
	\toprule
	Points & Criteria\\
	\midrule
	4 & Native support by tool\\
	3 & Extension exists, but installation necessary \\
	2 & Extension can be programmed in a popular programming language (R,Javascript,Java) \\
	1 & Extension can be programmed, but in a tool-specific programming language \\
	0 & No support by tool\\
	\bottomrule
	\end{tabu}
\end{table}


\subsection{Qlik Sense}
Qliktech was founded in 1993 with the goal to "mimic how the brain works."\cite{qlikHistory}. They offer five products(Qlik Sense, Qlik Sense Cloud, QlikView, QlikView NPrinting, Qlik DataMarket) and the Qlik Analytics platform. Qlik Sense 1.0 was released in September 2014 for visual analytics. 
It offers functions such as Smart Data Load which allows to load large data from different data sources.
\subsubsection*{Analytics}
For analytics Qlik Sense support \text{Visual Data Preparation}: showing data tables as bubbles and connecting them by dragging and dropping them. Moreover, the user can create calculated fields\cite{qlikCalculated}.

\subsubsection*{Visualization Techniques}
Qlik Sense offers 8 built-in visualization techniques: bar charts, line charts, pie charts, scatterplots, treemap, maps, combi charts and gauge charts. If an additional technique is wanted the user can build a visualization extension with \textit{javascript} and \textit{QEXT} files\cite{qlikWorkbench}. Qlik Sense provides an extension template which supports the user in writing its extensions. However, the user needs to know javascript and html\cite{qlikVisExtensions}. Moreover, the Qlik Community offers ...\todo{anzahl an visualisierungen einfügen}
\textbf{Aggregation}: In the field of aggregation Qlik Sense offers data aggregation for one chart type: the scatterplot. Hereby, large data is aggregated by aggregation markers (squares) which represents the data point density by the color. The darker the square the denser the data\cite{qlikScatter}. 

\subsubsection*{Interaction}
To allow a focus + context view Qlik Sense offers navigational maps\cite{beard1990navigational},.
As a built-in-function Qlik Sense offers a navigational slider which shows a miniature version of the whole data set\cite{beard1990navigational}. 
Filters can be applied by making selections in the visualization\cite{qlikSheet}, the time range can be limited by zooming inside the visualization\cite{qlikTime} and all views then are adapted to the current selection. Thus, Qlik Sense offers Brushing + Linking. An edditional linking-feature are \textit{master items}\cite{qlikChangeData}, which allow the user to change properties for all master items at once.
For details the user can search Qlik Sense with Smart Search in which the dimensions, measures and metadata is searched and visualizations ,tables and KPIs are displayed\cite{qlikSmart}.  

\subsection{Power BI}
Microsoft Power Bi came alive in ...\todo{datum einfügen}. It offers 15 different visualization techniques. 

\subsubsection*{Visualization Techniques}
Microsoft Power BI offers 8 built-in visualization techniques: 15 different visualization techniques. Moreover, the Power BI visuals gallery offers 75 visualization apps. To build custom visualization apps the user needs to write \textit{TypeScript} or \textit{R}.
\textbf{Aggregation}: In the field of aggregation Qlik Sense offers data aggregation for one chart type: the scatterplot. Hereby, large data is aggregated by aggregation markers (squares) which represents the data point density by the color. The darker the square the denser the data\cite{qlikScatter}. 
\textbf{Aggregation}: Power BI offers one way to aggregate: calculated fields.
\subsubsection*{Interaction}
With cross-highlighting Power BI included brushing and linking in the tool\cite{powerbiInteract}, it offers filter functions. The focus mode enables the user to have a detailed view on the visualization. In focus mode the visualization will expand to full screen.  

\chapter{Investigation of Visualization Tools using a Feature Classification Scheme}
\label{chap:Tools}

In this section we analyze selected visualization tools and their ability to visualize large time-oriented data. Therefore, the database metrics were compared for each tool. Database metrics is one part of the definition of visual scalability and describe how tools scale to large data sets. A detailed discussion can be found at section \ref{databasemetrics}. Moreover, we developed a classification scheme which is based on the collected success criteria of chapter \ref{chap:BIV} and ranked the tools according to this classification scheme. \\*

\section{Selection of Tools}\label{tool:selection}
The tools were selected based on their relevance in business. Nowadays, businesses strive to do self-service data science\cite{Russom2011,Parenteau2016,visualization2012making,curran2005self}. Therefore, they use self-service tools which are characterized by a graphical user interface (GUI), low prior programming skills and the universal use. The GUI enables non computer scientists to analyze data by the help of the human visual system. According to ITCentralStation and the Magic Quadrant for Business Intelligence and Analytics Platforms 2017 \cite{ITCentralStation, Sallam2017} and  Qlik, Tableau and Microsoft are the leading visionaries of BI Vendors. All of these tools proclaim to support the user in data visualization, self-service data science and to be easy to use. Thus, Qlik, Tableau and Microsoft Power BI are popular visualization tools (\ref{tools}). Their strength lies in the user support and their universality. Nevertheless, each tool is fee-based which results in investment costs. \\
Commercial software tends to need more time for the development and integration of advanced visualization for large data\cite{Zhanga, Simon2014}. Therefore, we added an open-source visualization tool to our selection. d3.js is a well-known choice for visualization in the visualization community as it is a free open-source JavaScript library and offers a wide range of visualization possibilities. \\
Thus, the selection of tools are Qlik Sense, Tableau, Microsoft Power Bi and d3.js.

%Nevertheless, market relevance reports from research and advisory companies such as Gartner, Forrester, Barc usually do not publish detailed scoring models.  Thus, the ranking might not be appropriate to our needs.


% \iffalse
% ADV in business context requires software that is able to scale visualization in an "effective manner"\cite{Russom2011}. Offering ADV techniques, parameterization, interaction and analytical methods such as data abstraction\cite{Tegarden1999,Aigner2011,Eick2002,Zhanga} are core functions of ADV software. \\*
% \textbf{The Role of APIs}
% Commercial software tends to need more time for the development and integration of advanced visualization for large data\cite{Zhanga, Simon2014}. To bridge the gap, vendors started to offer a bunch of APIs to expand the visualization functions. 
% % data load for BigData: 
% \textbf{Software not included in this work}
% As the goal of this work is to compare visualization tools in business we only consider software which is 
% \begin{enumerate}
%     \item generic: not specialized to one domain
%     \item integrates visualization features
% \end{enumerate}

% Furthermore, the software needs visualization features, the ability to present time-dependent data. Software with one of the following items is intentionally not considered: 
% \begin{enumerate}
%     \item Software that only presents one-dimensional data. 
%     \item Software that is specialized to data mining.
% \end{enumerate}
% \fi

\textbf{Qlik Sense: }
Qlik Sense is the self-service product of Qliktech. Qliktech was founded in 1993 with the goal to "mimic how the brain works."\cite{qlikHistory}. They offer five products(QS, QS Cloud, QlikView, QlikView NPrinting, Qlik DataMarket) and the Qlik Analytics platform. QS 1.0 was released in September 2014 for visual analytics. 
Self-Service data visualization describes the approach to encourage a broad public to do data analysis with easy-to-use tools.\\
\textbf{Power BI: }
Microsoft Power BI came alive in September 2013. It is divided in the three services Power BI Mobile, Power BI Desktop and Power BI service. Power BI Mobile access reports by a portable device. Power BI Desktop is a business analytics suite for creating visualizations and reports and Power BI service publishes reports. We are concentrating on Power BI Desktop.\\
\textbf{Tableau: }
Tableau was founded in 2003 out of a university project. Tableau sells three main products: Tableau Desktop, Tableau Public and Tableau Server.\\ 
\textbf{d3.js: }
d3.js is a JavaScript library used for visualization. It visualizes data based on SVG, HTML5 and CSS and binds data to existing web elements in alignment with the Document Object Model (DOM).  Data Handling is managed by the underlying data source. Data modeling can be handled by other JavaScript libraries such as node.js. A good overview how d3.js works is given in \cite{Meeks}. 
We chose d3.js as it is a data visualization tool with interaction. As it is free it becomes an alternative to commercial tools.\\


\section{Scalability of visualization tools}\label{tool:scalability}
Visual scalability of visualization tools is measured by the \textit{database metrics} and the \textit{visualization characteristics}. Moreover, as discussed in \ref{chap:BIV} scalability is enhanced by analytical methods and interaction techniques. Thus, the scalability of visualization tools will be compared by determining database metrics, visualization characteristics, analytical and interaction abilities for each tool.

\subsection{Database metrics}
Database metrics are defined as the size of the database which can be handled by tools\cite{Eick2002}. One possibility to measure the database size are the number of lines which can be loaded into the tool. Another approach to measure database metrics are the maximum number of rows which can be loaded into a visualization.
\par
While \textbf{d3.js}, \textbf{Power BI} and \textbf{Tableau} have no limitations how many data rows can be loaded, \textbf{QS} inherits the data load limitations of Qlik View: A QS document cannot have more than 2,147,483,648 distinct values in one field. This limitation still allows Qlik to handle large and huge data sets according to our definition. Moreover, 2 billion data points exceed the limit of screen pixels. But nevertheless, QS is outperformed by d3.js, Power BI and Tableau in this particular case.
\par
In context of visualization the number of rows which are loaded into a visualization is a determining factor for scalability. For large data the user expects to load all data he wants the tool to load. However, some tools limit the initial number of rows and the user has to write additional code. This effects the easy-of-use negatively.
QS limits the initial fetch to 10.000 but gives the opportunity to fetch more data if needed. \textbf{d3.js}, \textbf{Power BI} and \textbf{Tableau} currently have no limit for rows in visualization.
Again QS stays behind d3.js, Power BI and Tableau.

% Ranking for Database Metrics
\begin{table}[H]

    \begin{tabular}{|l| l l l l l|}
        \hline
        \multicolumn{2}{|c}{}   & d3.js  & QS  & Power BI & Tableau\\\hline
        \multirow{2}*{Database Metrics}
        & Maximum \# of rows in tool                & 1 & 4 & 1 & 1\\
        & Maximum \# of rows in visualization       & 1 & 4 & 1 & 1\\
        \hline
        \multicolumn{2}{|c}{}   & \textbf{1}    & \textbf{4}  & \textbf{1} & \textbf{1}\\
        \hline
    \end{tabular}
    \caption{Tool Ranking for criterion \textit{Database Metrics}}
    \end{table}

\subsection*{More than database metrics}
With the current development in technology, data management shifted from importing csv-files or Excel Spreadsheets to working with technologies such as clouds or databases. Thus, the maximum number of rows is no appropriate measure for database metrics anymore. Instead of measuring the loading time of data rows the connectivity to large scale cluster is the decisive factor. Most of the visualization tools today provide data engines, an underlying software component, which manages the data. Thus, the tool performance depends on the following factors:
\begin{enumerate}
    \item The underlying engine
    \item The connectivity to Big Data Technologies
\end{enumerate}
Other limiting factors are connection to multiple data sources, the perceived performance by the user and hardware resources.  Out of these factors visualization tools can influence the underlying engine, the connection, incremental loading and the connectivity to Big Data technologies. As this work focuses on the frontend perspective of scalability to large data sets the connectivity to Big Data Technology goes beyond the scope of this work. Here, we are shortly discussing the tool's underlying engine to make clear how the definition of database metrics needs to be adjusted. 

\textbf{The underlying Engine: In-memory versus Live Connection}
The architecture of engines in visualization tools can have the two forms: in-memory and live connection.
In-memory techniques store their data inside the RAM while live connections work directly on the database. 
With large data sets in-memory technologies might not be feasible. Even though working in-memory is faster than working directly at the database. With a smaller subset of large databases working in-memory might be the better option. 
\textbf{QS} data engine (QIX Engine) and \textbf{Power BI} use in-memory columnbased technology. While the data engine processes the calculation the RAM may temporarily be allocated. Thus, QS is limited by the primary memory of the computer. \textbf{Power BI} extends also to live connections to clouds similar to \textbf{Tableau}. Both tools offer in-memory technology as well as live connection. Tableau promotes the use of live connections instead of working in-memory even though working in-memory might be faster for small sets. Therefore, Tableau offers data extracts which can be created to work in-memory. The limits for data extracts are not published but Tableau Public, the free version of Tableau, recently extended the limit of 1 mio. rows to 10 mio. rows in-memory. 
d3.js is build to manage the visual frontend for visualizations. The data loading is outsourced to the backend. Therefore, d3.js can connect with any backend which implements the REST API. Thus, d3.js can not be compared with QS, Power BI and Tableau in this context. 
In summery, QS only offers in-memory and thus, has difficulties to work live on large data sets. Tableau and Power BI outperform QS in the criterion of database metrics.\\*
\textbf{Further Aspects: }
Scalability of tools can also be measured in terms of users and delivery. But this goes beyond our scope.


\subsection{Visualization Characteristics}

\subsubsection{The classification scheme}\label{tool:classification}\\*
The tools basis of assessment concerning \textit{visualization characteristics} is a classification scheme based on the success criteria of section \ref{success}. These aspects are assessed according to two business relevant criteria for self-service tools: the criterion of completeness and the criterion of required programming skills. The factor \textit{completeness} serves as indicator how many features can the business user achieve with this tool. Completeness in aggregation metaphors displays how many visualization techniques include aggregation metaphors. Yet, completeness is difficult to measure. Thus, we compare completeness by relative comparison.
\begin{table}[H]
	\caption[Tool Completeness]{Criteria Completeness: extend to which assessed aspect is implemented in tool}
	\label{programming-skills}
	\begin{tabu}{cl}
	\hline
	Points & Criteria\\
	\hline
	0 & Not existing\\
	1 & partially implemented \\
	2 & fully implemented \\
	\hline
	\end{tabu}
\end{table}

The consideration of programming skills in business is important as the standard business user of the visualization tool may have few programming knowledge (compare \ref{tasks}). Moreover, programming skills are connected with investment costs. The higher the required programming skills the higher are the investment costs. Programming skills are measured by the required user knowledge to achieve a feature. Least effort is needed when the tool offers the feature automatically (4). The next step is to embed a feature via drag and drop (3). This step still does not require any programming skill. On the next level programming knowledge is required but in a popular programming language (2). A popular programming language such as R, Java, JavaScript, python, C can used in other environments as well which increases the probability that the user knows the language. The difference from Step 2 to Step 3 is much larger than the difference from Step 3 to Step 4. Thus, the ranking is not linear but ordinal. The most complex level is when a feature can only be implemented by a tool-specific programming language (1). Here, we assume that a tool-specific language requires additional training effort. 
\begin{table}[H]
	\centering
	\caption[Programming-Skills for Tools ]{Criteria Required Programming-Skills to use the assessed aspect}
	\label{programming-skills}
	\begin{tabu}{cl}
	\toprule
	Points & Criteria\\
	\midrule
	1 & Feature can be programmed, but in a tool-specific programming language\\
	2 & Feature can be programmed in a popular programming language (R,JavaScript,Java)\\
	3 & Available by drag and drop \\
	4 & Automatic support by tool\\
	\bottomrule
	\end{tabu}
\end{table}

The criteria completeness and programming skills have been combined in the tool criteria score (TCS). Each feature of the tool has been assigned a tuple consisting of (Programming Skills, Completeness). If the assessed aspect did not exist 0 was assigned.
Based on the assessment of different features  we draw our conclusion (\ref{conclusion}).


% \iffalse
% \subsection*{R}
% \textbf{Analytical Techniques}\\
% Out of the five tools R has the most extensive offer of analytical methods for time-series data. It is possible to detect patterns such as outliers with tsoutliers\cite{Chen1993} and clusters with tsclust\cite{Manso2015}, do advanced analytics with tswge\cite{tswge} and forecasting with zra\cite{zra}.
% \textbf{Visualization Techniques}\\
% R supports standard visualization such as histograms, line charts and scatter plots. Advanced visualization techniques for large data sets 
% \textbf{Aggregation}\\*
% In the maps-package R offers the possibility to adjust the resolution with the parameter resolution. Resolution 0 maps the whole database whereas a higher resolution collapes data points within the resolution to one single point. This option allows to aggregate data and to only show the perceptual important points (PIP).
% The bigvis and the hexbin package implement binning to condense large data sets\cite{Wickham2013}.
% \textbf{Pixel-oriented} Time Series are visualized with mvtsplot\cite{mvtsplot}. This package allows to compare multivariate time-series data.
% \textbf{Interaction Techniques}\\
% Plot\_ly()\cite{plotly} supports a brushing function with drill-down.
% Shiny supports panning and zooming as well as linking and brushing.
% Linked views are supported by plot\_ly() with and without shiny. 
% Fish-eye views are implemented in the fisheyeR() package. 
% The time range can be decreased by limiting the data range inside the code.
% Animation can be achieved by using plot\_ly() and ggploty(). Moreover, plot\_ly() offers linked animated views.
% \textit{Dygraphs} includes also navigational sliders called Range Selector, panning and zooming and brushing of data items.
% \fi

\subsubsection{Detailed Tool Description}\label{tool:tools}
Before the TCS is assigned each tool is analyzed in detail with regard to the success criteria. \\

\noindent \textbf{Qlik Sense}
\par
\textbf{Analytical Techniques}\\
QS integrates methods to reduce data. Often, the tool specific \href{https://help.qlik.com/en-US/sense/3.2/Subsystems/Hub/Content/ChartFunctions/SetAnalysis/set-analysis-expressions.htm}{set expressions (SE)} are required therefore. 
Horizontal data reduction is implemented in \href{https://help.qlik.com/en-US/sense/3.2/Content/Videos/Videos-dimensions-limitations.htm}{\textit{Dimension Limitations}}. The user can decide upon the number of columns which are displayed. 
Dimensions are aggregated with \href{https://help.qlik.com/en-US/sense/3.2/Subsystems/Hub/Content/Dimensions/calculated-dimensions.htm}{\textit{calculated dimensions}}. Therefore, one or more dimensions are combined and saved as a new dimension.
Data reduction is part of QS Server. With \href{https://help.qlik.com/en-US/sense/2.1/Subsystems/Hub/Content/Scripting/Security/dynamic-data-reduction.htm}{\textit{dynamic data reduction (DDR)}} rows can be hidden for a group of users. Data rows can be reduced using SQL inside the backend. Thus, data size is decreased. Internally, .QVD files compress data. 
At the frontend, the user can apply \href{https://help.qlik.com/en-US/sense/2.1/Subsystems/Hub/Content/Visualizations/FilterPane/filter-pane.htm}{\textit{filters}}. QS \href{https://help.qlik.com/en-US/sense/2.1/Content/Videos/Videos-assoc-selection-model.htm}{\textit{standard selection mode}} includes brushing \& linking so that the selections of one visualization are automatically copied to all other visualizations. Additionally, the user can drag and drop custom filters inside the dashboard. 
Data clustering in QS is implemented by \href{https://help.qlik.com/en-US/sense/3.2/Subsystems/Hub/Content/Scripting/AggregationFunctions/aggregation-functions.htm}{\textit{aggregation-functions}}. These use SE and range from basic (min,avg,max) to advanced statistical functions(linear regression, correlation). With these aggregation functions QS offers some analytical methods. Yet, QS has no integration to an analytical program.
\par

\textbf{Visualization Techniques}\\
QS offers \href{https://help.qlik.com/en-US/sense/2.1/Subsystems/Hub/Content/Visualizations/visualizations.htm}{8 built-in visualization techniques}: bar charts, line charts, pie charts, scatter plots, treemap, maps, combi charts and gauge charts. If an additional technique is wanted the user can either install one of the community's self-made extensions or he can build his own \href{https://help.qlik.com/en-US/sense-developer/3.2/Subsystems/Extensions/Content/extensions-getting-started.htm}{\textit{visualization extension}} with \textit{JavaScript} and \textit{QEXT} files\cite{qlikWorkbench}. QS provides an extension template which supports the user in writing its extensions. Moreover, Qlik provides 20 high-level-APIs which supports the user in writing a custom extension. However, the user needs to know JavaScript, html\cite{qlikVisExtensions}, as well as QS own QEXT-language. \\*
Out of the QS standard repertoire non of the visualization technique corresponds to the studied visualization techniques of chapter \ref{chap:BIV}. With d3.js it would be possible to build these visualization techniques and integrate them in QS. 
The embedding of JavaScript also allows \textbf{aggregation} in terms of multi-resolution and the use of aggregation markers. However, these advanced metaphors have to be implemented for each technique and thus are only available for some techniques. Moreover, the implementation requires coding-skills. 

In the field of built-in-aggregation markers QS offers data aggregation for one chart type: the scatter plot. Hereby, large data is aggregated by aggregation markers. When the scatter plot is shown at an overview level accumulated data points are represented by squares. Data density is mapped to the color attribute.  The darker the square the denser the data\cite{qlikScatter}. The so called \href{https://help.qlik.com/en-US/sense/2.1/Subsystems/Hub/Content/Visualizations/scatter plot/scatter-plot.htm}{\textit{Smart Data compression}(SDC)} is one implementation for the aggregation of large data amounts. However, SDC is only available for one technique and that is why QS still stays behind the present day requirements.


\begin{figure}[H]
    \centering
    \subfloat[QS]{\includegraphics[width=6cm]{src/images/SmartDataCompression}}
    \caption{Smart Data Compression (left): Overview level which shows aggregated data points by squares and color and changes the shape if zoomed in}
    \label{fig:smartdatacompression}
\end{figure}

\begin{figure}[H]
    \centering
    \subfloat[Sparse Area]{\includegraphics[width=6cm]{src/images/SmartDataCompressionI}}
    \qquad
    \subfloat[Dense Area]{\includegraphics[width=6cm]{src/images/SmartDataCompressionII}}
    \caption{Smart Data Compression: Detail level}
    \label{fig:smartdatacompression}
\end{figure}

\textbf{Interaction Techniques}\\
In terms of interaction techniques QS offers drill-down and navigation techniques. Zooming and filtering are possible for all visualizations. If the zoom is active mini charts (MiC) appear which are one implementation of navigational maps\cite{beard1990navigational}. Mini Charts come as a navigational slider in which a miniature version of the whole data set\cite{beard1990navigational} is shown. 
Filters can be applied by making selections or dragging a filter inside the visualization\cite{qlikSheet}. All views then are adapted to the current selection. Thus, QS offers Brushing + Linking. For details the user can search QS with \href{https://help.qlik.com/en-US/sense/2.1/Content/Videos/Videos-global-smart-search.htm}{\textit{Smart Search}} in which the dimensions, measures and metadata is searched and visualizations, tables and KPIs are displayed\cite{qlikSmart}. Distortion techniques are not offered but can be implemented with extensions. 
\par
\noindent \textbf{Tableau}
\par
\textbf{Analytical Techniques}\\
Tableau's strength are easy-to-use analytical functions. On one hand, Tableau offers build-in modelling functions: prediction, trend line, cluster, average and median. On the other hand data reduction are offered and R scripts can be loaded into Tableau. Horizontal and vertical reduction is implemented by \href{http://onlinehelp.tableau.com/current/pro/desktop/en-us/extracting_data.html}{\textit{Data Extracts (DE)}}. These can be build based on the loaded data connection and remove reduce the data by limiting the loaded data. Dimensions can be reduced by hiding columns while creating DE. Only visible columns are then loaded into the dashboard. Dimensions are aggregated similar to QS by \href{http://onlinehelp.tableau.com/current/pro/desktop/en-us/calculations_calculatedfields.html}{\textit{calculated fields}}. Moreover,  \href{http://onlinehelp.tableau.com/current/pro/desktop/en-us/calculations_calculatedfields_aggregate.html}{\textit{aggregated calculations}} reduce data vertically by clustering them. Another implementation of vertical data reduction are  \href{http://kb.tableau.com/articles/howto/adding-filters-to-dashboards}{\textit{filter}} which can be added in the frontend. Tableaus analytical capabilities can be enhanced by integrating R which offers additional data reduction capabilities such as PCA or SOM.
\par
\textbf{Visualization Techniques}\\
Tableau offers 22 built-in visualization techniques: heat maps, symbol maps, stacked bars, pie charts, horizontal bars, side-by-side bars, treemaps, circle views, side-by-side circles, continuous lines, discrete lines, dual lines, area charts,  discrete area charts, dual combination, scatter plots, histogram, box-and-whisker plots, Gantt chart, bullet graphs and packed bubbles.
Visualization extensions are not possible even though Tableau has a \href{https://www.google.de/search?client=safari&rls=en&q=tableau+javascript+api&ie=UTF-8&oe=UTF-8&gfe_rd=cr&ei=2oLOWKveK5LZ8AeXl4bIBw}{JavaScript API}. This API allows the integration of a Tableau dashboard into a web page, but is not built for writing JavaScript extensions for Tableau. Besides, advanced metahpors such as multi-resolution, data abstraction or aggregation markers are not supported. 
In sum, Tableau is a well-known choice in the visualization community. But it does not support large scale visualization features. 
\par

\textbf{Interaction Techniques}\\
Speaking of drill-down techniques Tableau offers automatically zooming and  \hyperlink{http://kb.tableau.com/articles/howto/adding-filters-to-dashboards}{filtering}.
Regarding navigation techniques Tableau integrates \href{https://www.tableau.com/de-de/whitepapers/enhancing-visual-analysis-linking-multiple-views-data}{\textit{multiple linked views}}.
Distortion techniques, search and navigational sliders are not supported. 
\par

\noindent \textbf{Power BI}
\par
\textbf{Analytical Techniques}\\
Data reduction in Power BI is implemented similar to QS and Tableau. Dimensions can be aggregated with 
Horizontal data reduction implemented by  \href{https://powerbi.microsoft.com/en-us/documentation/powerbi-desktop-tutorial-create-calculated-columns/}{\textit{calculated columns}}. Vertically, data can be reduced both in the frontend and in the frontend. With \href{https://community.powerbi.com/t5/Desktop/How-to-reduce-the-amount-of-data-that-is-loaded-into-my-Power-BI/td-p/54112}{\textit{views}} on the database (V) and joins data extracts are created. Moreover, data sets can be extracted (DE) in the frontend with \href{https://Power BI.microsoft.com/de-de/blog/power-bi-desktop-october-feature-summary/#grouping}{\textit{Inclusion/Exclusion}} of data points. Moreover, \href{https://powerbi.microsoft.com/en-us/documentation/powerbi-service-add-a-filter-to-a-report/}{\textit{filters}} can be added and  \href{https://powerbi.microsoft.com/en-us/documentation/powerbi-service-aggregates/}{\textit{aggregates}} build.
Besides the own core functions for data reduction Power BI is connected to analytical programs such as \href{https://Power BI.microsoft.com/de-de/blog/power-bi-desktop-october-feature-summary/#grouping}{\textit{R, Mixpanel or comScore.}}
\par

\textbf{Visualization Techniques}\\
Microsoft Power BI offers 8 built-in \href{https://powerbi.microsoft.com/en-us/documentation/powerbi-service-visualization-types-for-reports-and-q-and-a/}{\textit{visualization techniques}}, 15 available techniques at the MarketPlace and 75 visualization apps in the visuals gallery. Additionally, the user can build custom visualization apps by writing \href{https://powerbi.microsoft.com/en-us/documentation/powerbi-custom-visuals-getting-started-with-developer-tools/}{\textit{TypeScript}} or \href{https://powerbi.microsoft.com/en-us/guided-learning/powerbi-learning-3-11h-r-visual-integration/}{\textit{R}}. Out of the 98 techniques none of them implements any discussed techniques of chapter \ref{chap:BIV}. So far advanced Metaphors are not supported\cite{Amanda}. One approach to aggregation markers are \hyperlink{https://Power BI.microsoft.com/de-de/blog/power-bi-desktop-october-feature-summary/#grouping}{\textit{Groups}} which can be combined with drill-down options.
\par

\textbf{Interaction Techniques}\\
Power BI embeds zoom and filter functions. The \textit{focus mode} expands one visualization to full screen and thus, enables the user to have a detailed view on the visualization. Thus, Power BI implements an overview + detail view. 
With \href{https://powerbi.microsoft.com/en-us/documentation/powerbi-service-about-filters-and-highlighting-in-reports/}{\textit{highlighting}} Power BI includes brushing and linking\cite{Power BIInteract}. Therefore, the user can control which windows are connected. The app \textit{Advanced Time Slicer} is one implementation of the navigation technique navigational map for time-oriented data. Navigation in Power BI is also realized by the search mechanism Power Q\&A which answers NLP-question regarding the data set. Power Q&A includes filtering with the keywords WHERE, AFTER, BEFORE, BETWEEN, WITH or by naming the date as well. Distortion techniques are not supported. They can be embedded with visualization extensions (E) written in TypeScript.

\par
\noindent \textbf{d3.js}
\par
\textbf{Analytical Techniques}\\
While d3.js is only designed to visualize data it does not inherently offer the ability to analyze data. Therefore, a number of JavaScript libraries (L) exist which can be combined with d3.js. Some examples are simplestatistics.js, regression.js, node.js with the packages data-reduction, ml-pca or dimensionality-reduction. As JavaScript allows the integration of libraries there is no limit for analytical functions. Thus, d3.js does not offer any data reduction techniques automatically. Besides libraries d3.js can be used together with any REST API compatible backend and the data reduction abilities then depend on the backend.
Regarding data abstraction d3.js offers abstraction functions the frontend. Some examples are d3.hexbin which allows binning, simplify.js for data abstraction or clusterfck for clustering. Simplify.js for example reduces data points on polylines while maintaining the characteristic shape of the polyline. 
\par

\textbf{Visualization Techniques}\\
d3.js assigns data attributes to graphical attributes. Thus, different way exist to visualize the same visualization technique. By writing JavaScript code any visualization technique and any advanced metaphor can be implemented. Currently, various implementations are \href{https://github.com/d3/d3/wiki/Gallery}{\textit{available}} which helps the user to create its own visualizations. For aggregation metaphors \href{http://bl.ocks.org/gisminister/10001728}{\textit{marker clustering}} is one example. Moreover, tutorials exist which describe how marker clustering can be implemented with  \href{https://www.phase2technology.com/blog/using-d3-quadtrees-to-power-an-interactive-map-for-bonnier-corporation/}{quadtrees}\cite{Morrison2014}.
\par

\textbf{Interaction Techniques}\\
d3.js allows to integrate various interaction techniques such as Zooming, Filtering or Linking \& Brushing. Distortion techniques can be achieved with the d3 plugin \hyperlink{https://bost.ocks.org/mike/fisheye/}{\textit{Fisheye Distortion (FD)}}\cite{Bostock2012} which allows circular, linear and logarithmic distortion.
Perspective walls can be implemented with \hyperlink{https://bl.ocks.org/mbostock/10571478}{\textit{Perspective Transformation (PT) }}\cite{Bostock2017}.\\

\subsubsection{Ranking of Tools}
All tool-features are listed in table \ref{table:features} and in table \ref{table:TCS} the TCS is derived. Then, the tools are ranked from the first place (1) to the last place(4). The first place is characterized by offering most of the features for large scale visualization relative to the other tools. Nevertheless, the first place is no optimum. 

\begin{table}[H]

    \begin{tabular}{|l| l l l l l|}
        \hline
        \multicolumn{2}{|c}{}   & d3.js  & QS  & Power BI & Tableau\\\hline
        \multirow{9}*{Analytics}
        & \multicolumn{5}{l|}{\cellcolor{gray!30} Horizontal Data Reduction}\\\cline{2-6}
        & Data can be reduced to $k$ dimensions & L & DL & - & H \\
        & Dimensions can be aggregated & L & CD & CC & CF\\ \cline{2-6}
        & \multicolumn{5}{l|}{\cellcolor{gray!30}Vertical Data Reduction}\\\cline{2-6} 
        & Omitting & L & DDR & J,V & DE,E\\
        & Filtering & L & DQ,B & DQ & DE,DQ,B\\
        & Removal & L & O & DE & DE \\
        & Abstraction & L & - & - & - \\
        & Aggregation & L & A & A & A \\\cline{2-6}
        &\rowcolor{gray!30}  Data Modeling & L  & -    & T,MM    & T, MM, F \\\cline{2-6}
        &\rowcolor{gray!30}  Pattern Search& L  & -    & -       & - \\
        \hline
        \multirow{3}*{Visualization}
        & Offers ADV                & P     & E    & E &-   \\
        & Multi-Resolution          & P     & E    & E & -  \\
        & Aggregation Markers       & MC    & SDC  & E & -  \\
        
        \hline
        \multirow{11}*{Interaction}
        
        & \multicolumn{5}{l|}{\cellcolor{gray!30}Drill-Down Functions}\\\cline{2-6}
        & Filter    & P & N & N & N \\
        & Zoom      & P & N & N & N \\ \cline{2-6}
        
        & \multicolumn{5}{l|}{\cellcolor{gray!30}Distortion Techniques}\\\cline{2-6}
        & Graphical Fish-eye    & \hyperlink{https://bost.ocks.org/mike/fisheye/}{FD}\cite{Bostock2012}       & E  & E  & - \\
        & Bifocal-Display       & \hyperlink{https://bost.ocks.org/mike/fisheye/}{FD}\cite{Bostock2012}       & E  & E  & - \\
        & Perspective Walls     & PT & E & E & - \\ \cline{2-6}
        
        & \multicolumn{5}{l|}{\cellcolor{gray!30}Navigation Techniques}\\\cline{2-6}
        & Brushing \& Linking   & P & N & N & N \\
        & Search                &  L & \hyperlink{https://help.qlik.com/en-US/sense/2.2/Subsystems/Hub/Content/Search/search-tool.htm}{SS}& \hyperlink{https://powerbi.microsoft.com/en-us/documentation/powerbi-service-q-and-a/}{Q\&A}& - \\
        & Navigational Maps     & P & \hyperlink{https://help.qlik.com/en-US/sense/1.1/Subsystems/Hub/Content/Visualizations/BarChart/BarChart.htm}{MiC}  & -           & -\\
        \hline
    \end{tabular}
    \caption{Tool implementations of success criteria}
    \label{table:features}
    \end{table}
    
    Analytics\\*
    L = Library, B = Brushing, CC = Calculated Columns, CD = Calculated Dimensions, CF = Calculated Fields, DA = Data Abstraction, DL = Data Limitations, DDR = Dynamic Data Reduction, DE = Data Extraction, DQ = Dynamic Query Filtering, E = Extensions, F = Forecasting, H = Hide Columns in Data Extract, J = Joins, MM = Min-Max-Function, O = Omit row in SQL-Script, T = Trendline, V = Views, - = not existing
    \par 
    Visualization\\*
    P = Programmable, E = Extensions, MC = Marker Clustering, SDC = Smart Data Compression, - = not existing
    \par
    Interaction\\*
    E = Extensions, FD = Fisheye Distortion, MiC = Mini Charts, N = Natively Integrated, P = Programmable, PT = Perspective Tranformation, Q\&A = Question \& Answer,  SS = Smart Search, - = not existing

\begin{table}[H]

    \begin{tabular}{|l| l l l l l|}
        \hline
        \multicolumn{2}{|c}{}   & d3.js  & QS  & Power BI & Tableau\\\hline
        \multirow{9}*{Analytics}
        & \multicolumn{5}{l|}{\cellcolor{gray!30} Horizontal Data Reduction}\\\cline{2-6}
        & Data can be reduced to $k$ dimensions & 2,2 & 4,1 & 2,2 & 4,2 \\  
        & Dimensions can be aggregated & 2,2 & 1,2 & 1,2 & 4,2  \\\cline{2-6}
        & \multicolumn{5}{l|}{\cellcolor{gray!30}Vertical Data Reduction}\\\cline{2-6}
        & Omitting               & 2,2 & 2,2 & 2,2 & 4,2 \\
        & Filter    & 2,2 & 3,2/4,2 & 3,2 & 3,2/4,2\\
        & Removal   & 2,2 & 2,2 & 4,2 & 4,2 \\
        & Abstraction           & 2,2 & 0 & 0 & 0\\
        & Aggregation           & 2,2 & 1,2 & 3,1 & 3,2 \\\cline{2-6}
        &\rowcolor{gray!30} Data Modeling  & 2,2 & 1,1 & 3,1 & 3,1 \\\cline{2-6}
        &\rowcolor{gray!30} Pattern Search & 2,2 & 0 &  0  & 0\\
        \hline
        \multirow{3}*{Visualization}
        & Offers ADV            &   2,2  &  1,2 & 1,2 & 0  \\
        & Aggregation Markers   &   2,2  &  1,1/4,1 & 1,1 &  0 \\
        & Multi-Resolution      &   2,2  &  1,1 & 1,1 & 0  \\
        
        \hline
        \multirow{11}*{Interaction}
        
        & \rowcolor{gray!30}Drill-Down Functions & & & &\\\cline{2-6}
        & Filter  & 2,2 & 4,2 & 4,2 & 4,2 \\ 
        & Zoom    & 2,2 & 4,2 & 4,2 & 4,2 \\\cline{2-6}
        
        & \rowcolor{gray!30}Distortion Techniques & & & &\\\cline{2-6}
        & Graphical Fish-eye    & 2,2 & 1,1 & 1,1 & 0 \\
        & Bifocal-Display       & 2,2 & 1,1 & 1,1 & 0 \\
        & Perspective Walls     & 2,2 & 1,1 & 1,1 & 0 \\ \cline{2-6}
        
        & \rowcolor{gray!30}Navigation Techniques & & & &\\\cline{2-6}
        & Brushing \& Linking   & 2,2 & 4,2 & 4,2 & 4,2\\
        & Search                & 2,2 & 4,2 & 1,1 & 0 \\
        & Navigational Maps     & 2,2 & 4,2 & 2,1 & 0 \\
        \hline
    \end{tabular}
    \caption{TCS for tools}
    \label{table:TCS}
    \end{table}

% Ranking for Programming Skills
\begin{table}[H]

    \begin{tabular}{|l| l l l l l|}
        \hline
        \multicolumn{2}{|c}{}   & d3.js  & QS  & Power BI & Tableau\\\hline
        \multirow{5}*{Analytics}
        & \rowcolor{gray!30}            & \textbf{4} & \textbf{3} & \textbf{2} & \textbf{1}\\\cline{2-6}
        & Horizontal Data Reduction     & 3 & 2 & 4 & 1\\
        & Vertical Data Reduction       & 4 & 3 & 2 & 1\\
        & Data Modeling                 & 3 & 4 & 1 & 1\\
        & Pattern Search                & 1 & 4 & 4 & 4\\
        \hline
        \multirow{3}*{Visualization}
        & \rowcolor{gray!30}    & \textbf{2}    & \textbf{1} & \textbf{3} & -\\\cline{2-6}
        & Offers ADV            & 1 & 2 & 2 & 4 \\
        & Aggregation Markers   & 2 & 1 & 3 & 4 \\
        & Multi-Resolution      & 1 & 2 & 2 & 4  \\
        
        \hline
        \multirow{3}*{Interaction}
         & \rowcolor{gray!30}   & \textbf{3}    & \textbf{1} & \textbf{2} & \textbf{4}\\\cline{2-6}
        & Drill-Down Functions  & 4 & 1 & 1 & 1    \\
        & Distortion Techniques & 1 & 2 & 2 & -    \\        
        & Navigation Techniques & 3 & 1 & 2 & 4    \\
        \hline
        \hline
        \multicolumn{2}{|c}{}   & \textbf{4}    & \textbf{2}  & \textbf{3} & \textbf{1}\\
        \hline
    \end{tabular}
    \caption{Tool Ranking for criteria \textit{Programming Skills}}
    \end{table}

% Ranking for Completeness
\begin{table}[H]

    \begin{tabular}{|l| l l l l l|}
        \hline
        \multicolumn{2}{|c}{}   & d3.js  & QS  & Power BI & Tableau\\\hline
        \multirow{5}*{Analytics}
        & \rowcolor{gray!30}            & \textbf{1} & \textbf{3} & \textbf{4} & \textbf{2}\\\cline{2-6}
        & Horizontal Data Reduction     & 1 & 4 & 1 & 1\\
        & Vertical Data Reduction       & 1 & 2 & 4 & 2\\
        & Data Modeling                 & 1 & 4 & 4 & 4\\
        & Pattern Search                & 1 & 4 & 4 & 4\\
        \hline
        \multirow{3}*{Visualization}
        & \rowcolor{gray!30}            & \textbf{1} & \textbf{2} & \textbf{3} & \textbf{4}\\\cline{2-6}
        & Offers ADV            & 1 & 2 & 2 & 4 \\
        & Aggregation Markers   & 1 & 2 & 3 & 4 \\
        & Multi-Resolution      & 1 & 2 & 2 & 4  \\
        
        \hline
        \multirow{3}*{Interaction}
         & \rowcolor{gray!30}   & \textbf{1} & \textbf{2} & \textbf{3} & \textbf{4}\\\cline{2-6}
        & Drill-Down Functions  & 1 & 1 & 1 & 1    \\
        & Distortion Techniques & 1 & 2 & 2 & 4    \\        
        & Navigation Techniques & 1 & 1 & 3 & 4    \\
        \hline
        \hline
        \multicolumn{2}{|c}{}   & \textbf{1}  & \textbf{2}  & \textbf{3} & \textbf{4}\\
        \hline
    \end{tabular}
    \caption{Tool Ranking for criteria \textit{Completeness}}
    \end{table}

\newpage
\section{Conclusion}

The tool comparison demonstrates that there exists different strategies for visualizing  large data in visualization tools: the analytical strategy and the visualization/interaction strategy. Tableau's strength is the analytical data visualization with a focus of \textit{data preprocessing} which is shown in the \textit{ease-of-use} ranking. Tableau takes the first place and the second in the \textit{completeness} ranking as this tool offers a broad range of analytical functions automatically or the user can drag functions such as clustering and prediction inside the dashboard. Moreover, the integration of R exploits \textit{data reduction} techniques and time-oriented user tasks. Tableau  promotes the reduction of a data set before loading it into Tableau as some features are only available for reduced data extracts. These features are count distinct, offline access and incremental refresh. d3.js itself does not offer any analytical methods. But as d3.js is a JavaScript library it can be extended by other libraries which support data reduction, data modeling or pattern search. While comparing QS, Power BI, Tableau and d3.js one has to consider that JavaScript is a turing complete programming language. Thus, all success features defined in \ref{chap:BIV} can be programmed with JavaScript which explains the first rank of d3.js among all categories. Nevertheless, visualizing in d3.js requires programming-skills and time and thus, d3.js never took the first place in the \textit{programming-skills} ranking.
On the contrary QS' lacks in analytical options. While Tableau can integrate R-algorithms to reduce data, QS only can manipulate data with the QS specific set expressions. However, set expressions are limited in data reduction possibilities.\\
Power BI combines both analytical and visualization features. One can build visualizations with R and TypeScript, and also execute R scripts in Power BI. \\
Still, the support of analytic techniques for time-oriented user tasks is only partially implemented. Tableau and Power BI provide forecasting but analytical techniques for time-oriented such as temporal abstraction or pattern search are not provided by any tool at the moment.
\par

The Visualization strategy goes along with interaction as both consider the frontend for visualizations. QS pursues that strategy which is shown in first implementations of aggregation markers with Smart Data Compression and the extendability. Its strength is the JavaScript-interface in which new visualizations can be integrated into QS. \\
Similar to QS visualization extensions are supported by Power BI whereas Tableau's support of visualization techniques lacks of ADV visualizations. Besides the standard repertoire no extensions can be installed. Hence, visualization of many data items currently lead to clutter and disorientation. As d3.js is a JavaScript library any visualization and advanced visual metaphor can be created.
The tool comparison regarding visualization showed that ADV requires the use of programming languages. Current visualization tools offer a standard repertoire of visualizations. These techniques are easy to use as they apply drag and drop. Yet, none of the techniques for multivariate time-oriented data is implemented by Tableau, QS or Power BI. Therefore, extensions are required which are implemented in a programming language. Furthermore, the languages for visualization extensions include tool specific languages such as Type Script and QEXT.\\
\par
Large-scale interaction techniques show the following pattern. Either they are automatically integrated in tools such as drill-down functions or they are not implemented in any tool such as distortion functions. The only way to integrate distortion functions are via extensions. Then the particular interaction technique is only available for the respective visualization technique. Yet, interaction techniques should general be available for every visualization technique.
Regarding navigation techniques coordinated windows are supported by all tools while QS and Power BI are the only tool which implements search. QS then is the only tool which provides navigational maps.
In short, all tools lack in distortion techniques and Tableau and Power BI in navigation techniques. 
\par
In summary, currently there exists no all-in-one solution for the analysis of large time-oriented data in business. While d3.js any possibility regarding visualization and  interaction programming knowledge is required. However, this conflicts with the business approach to do self-service data science. When business decide to use tools such as Tableau, QS and Power BI they need to be aware about the limitations in displaying large data in an effective manner. 





\chapter{Future Work}
\label{Future Work}

\section{Limitations} \label{limitations}
In Chapter 3 we classified the visualization techniques according to their scalability. Therefore, we made assumptions such as the number of data rows and number of attributes \textit{based on the original paper} when they got published. We did not consider any updates and extensions of the respecting visualization technique. To give an example, the scalability of \textit{TimeWheel} was set on the 2D-TimeWheel although an extended 3D-Version exists. We decided to consider the visualization techniques mentioned in\cite{Aigner2011} as they are recommended for time-oriented data but extensions may have a better scalability.  
Moreover, we took the classification of Aigner et al.\cite{Aigner2011} regarding univariate and multivariate. If the judgement was wrong, our findings are also affected.
In Chapter 4 the success criteria score is based on the assumption that wrinting in a tool specific language is more difficult for a business user than writing in a known language such as Java, javascript or R. This assumption does not assume that a tool specific language might support the user better than only writing in a known language. This limitation has to be considered in the interpretation of our results.
\section{Future Research}


As business users should not be distracted by complex visualization systems \cite{Tegarden1999} evaluating the applicability for business of the proposed visualization techniques would be an important future work.


% Include more chapters here.

% End of main part
% ---------------------------------

% ---------------------------------
% Begin of appendix

\appendix

\printnoidxglossary[title={List of Abbreviations}]
% Appendix chapters are optional. Use it if you have very long tables or additional figures that
% do not belong to the main text.
\chapter{Appendix}
%\pagestyle{empty}
%\rotatebox{90}{
\begin{sidewaystable}
\centering
  \begin{tabular}{|l| l l l l l|}
        \hline
        \multicolumn{2}{|c}{}   & d3.js  & \gls{QS}    & Power BI & Tableau\\\hline
        \multirow{9}*{Analytics}
        & \multicolumn{5}{l|}{\cellcolor{gray!30} Horizontal Data Reduction}\\\cline{2-6}
        & \makecell[|l]{Data can be reduced \\to $k$ dimensions} & With Library & Dimension Limitations & - & \makecell[l]{Hide Columns in \\Data Extracts} \\
        & Dimensions can be aggregated & With Library & Calculated Dimensions & Calculated Columns & Calculated Fields\\ \cline{2-6}
        & \multicolumn{5}{l|}{\cellcolor{gray!30}Vertical Data Reduction}\\\cline{2-6} 
        & Omit rows in SQL-Script & With Library & Dynamic Data Reduction & \makecell[l]{Joins and Views \\on the database} & \makecell[l]{Data Extracts, \\with Extensions}\\
        & Filtering & With Library & \makecell[l]{Dynamic Queries, \\Brushing} & Dynamic Queries & \makecell[l]{Data Extracts, \\Dynamic Queries, \\Brushing}\\
        & Removal & With Library & Omit rows in SQL-Script & Data Extracts & Data Extracts \\
        & Abstraction & With Library & - & - & - \\
        & Aggregation & With Library & Aggregation & Aggregation & Aggregation \\\cline{2-6}
        \rowcolor{gray!30} \cellcolor{white} &  Data Modeling & With Library  & -    & Trends, Min/Max    & \makecell[l]{Trends, Min/Max, \\Forecast} \\\cline{2-6}
        \rowcolor{gray!30} \cellcolor{white} &  Pattern Search& With Library  & -    & -       & - \\
        \hline
        \multirow{3}*{Visualization}
        & Offers ADV                & Programmable     & with Extension    & with Extension &-   \\
        & Multi-Resolution          & Programmable     & with Extension    & with Extension & -  \\
        & Aggregation Markers       & Marker Clustering    & Smart Data Compression  & with Extension & -  \\
        
        \hline
        \multirow{11}*{Interaction}
        
        & \multicolumn{5}{l|}{\cellcolor{gray!30}Drill-Down Functions}\\\cline{2-6}
        & Filter    & Programmable & Native Integration & Native Integration & Native Integration \\
        & Zoom      & Programmable & Native Integration & Native Integration & Native Integration \\ \cline{2-6}
        
        & \multicolumn{5}{l|}{\cellcolor{gray!30}Distortion Techniques}\\\cline{2-6}
        & Graphical Fish-eye    & \hyperlink{https://bost.ocks.org/mike/fisheye/}{Fisheye Distortion}  & with Extension  & with Extension  & - \\
        & Bifocal-Display       & \hyperlink{https://bost.ocks.org/mike/fisheye/}{Fisheye Distortion} & with Extension  & with Extension  & - \\
        & Perspective Walls     & \makecell[l]{Perspective \\ Transformation} & with Extension & with Extension & - \\ \cline{2-6}
        
        & \multicolumn{5}{l|}{\cellcolor{gray!30}Navigation Techniques}\\\cline{2-6}
        & Brushing \& Linking   & Programmable & Native Integration & Native Integration & Native Integration \\
        & Search                &  with Library & \hyperlink{https://help.qlik.com/en-US/sense/2.2/Subsystems/Hub/Content/Search/search-tool.htm}{Smart Search}& \hyperlink{https://powerbi.microsoft.com/en-us/documentation/powerbi-service-q-and-a/}{Q+A}& - \\
        & Native Integration & Navigational Maps     & Programmable & \hyperlink{https://help.qlik.com/en-US/sense/1.1/Subsystems/Hub/Content/Visualizations/BarChart/BarChart.htm}{Mini Charts}  & -           \\
        \hline
    \end{tabular}
    \caption{Long version: Tool Implementations of Success Criteria}
    \label{table:long:features}
    \end{sidewaystable}
%}
\markboth{}{}

% Remove this from the final document
%\chapter{Checklist}
\label{chap:appendix:checklist}
Use the following list to check if you have followed the hints from this document in your work.
Refer to \cref{chap:introduction} for more information on the items.

\tabulinesep=2.5mm
% tabu allows to use relative width specifier: Use X[<ratio>] to specify the width of a column. In
% this example, the table is divided into 20 parts that are spread over the three columns.
\begin{longtabu} to 0.8\textwidth {rX}
% Begin header on first page
\caption[Formatting checklist]{The checklist for correctly formatted and prettier documents.}
\label{tab:checklist}
\\ \addlinespace
\endfirsthead
% End header on first page

% Begin header on consecutive pages
\caption[]{The checklist for correctly formatted and prettier documents, continued.}
\\ \addlinespace
\toprule
\endhead
% End header on consecutive pages

% Begin footer
\\ \addlinespace
\multicolumn{2}{c}{Table is continued on the next page.} \\
\endfoot
% End footer

% Begin footer on last page
\endlastfoot
% End footer on last page

% Begin content
\toprule
1. & Check for incorrect/missing citations (\emph{(?)}) or references (\emph{??}).\\
\midrule
2. & Remove all \LaTeX{} errors and underfull/overfull boxes.\\
\midrule
3. & Make sure to use a consistent encoding for your \TeX{} files, especially for special characters (ä, ö, ü, ß, \dots).\\
\midrule
4. & Format your \textsc{Bib}\TeX{} document: Check if the information is correct and complete for each entry. Pay notice to warnings when running  the \texttt{bibtex} command.\\
\midrule
5. & Add access dates to online sources in your bibliography (see \texttt{library.bib}) or in footnotes.\\
\midrule
6. & Run a spell checker over your document (included with some \LaTeX{} editors).\\
\midrule
7. & Place figures or tables in the \TeX{} file at the end of the paragraph you are referring to them in the text (\texttt{\textbackslash{}cref}).\\
\midrule
8. & Use the correct format for quotation marks (\emph{``x''} or \emph{\glqq{}x\grqq{}}).\\
\midrule
9. & Use the correct format for separating paragraphs (one empty line). Manual line breaks (\texttt{\textbackslash{}\textbackslash}) should be avoided.\\
\midrule
10. & Use the correct form of dashes (-, --, ---).\\
\midrule
11. & Name sources of images/figures you did not create yourself in the description (\emph{Source: [X]}).\\
\midrule
12. & Use the short form of captions for List of Figures, Tables, etc. (\texttt{caption[<short>]\{<long>\}}).\\
\midrule
13. & If you print your work double-sided (recommended for bachelor and master thesis) remove the \texttt{oneside} option from the document class.\\
\midrule
14. & Make sure to use high-quality figures and images that are readable both in the digital and the printed version.\\
\midrule
15. & If you include the declaration of honor do not forget to sign it.\\
\bottomrule
% End content
\end{longtabu}

\backmatter


% Fix for long URLs in bibliography
\sloppy
\bibliography{mendeley}
\fussy

\declarationofhonorchap

I declare that the work in this paper is completely my own work and that I have not used any other resources than the ones indicated. Any parts taken from other books, papers and authors have been indicated by giving credit to the author. All references have been clearly cited. This paper has not been presented to other examination offices. 

\vspace{1cm}
\noindent
\insertcitydate{Magdeburg}{\today}

\vspace{1.5cm}
\noindent
\insertauthor

% Consult your supervisor about the following declaration of assignment.
%\include{src/abtretungserklaerung}

% End of appendix
% ---------------------------------

\end{document}
